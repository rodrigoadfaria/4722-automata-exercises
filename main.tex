% --------------------------------------------------------------
% This is all preamble stuff that you don't have to worry about.
% Head down to where it says "Start here"
% --------------------------------------------------------------

\documentclass[12pt]{article}

\usepackage[latin1,utf8]{inputenc}
\usepackage[brazil]{babel}

\usepackage[margin=1in]{geometry}
\usepackage{amsmath,amsthm,amssymb}
\usepackage{clrscode3e} % for the pseudocode
\usepackage{graphicx}
\usepackage{float}
\usepackage{framed}
\usepackage{caption}
% for tables
\usepackage{booktabs}
\usepackage[table,xcdraw]{xcolor}
% for the graphs %
\usepackage{tikz}
\usetikzlibrary{trees}
\usetikzlibrary{shapes,snakes}
\usepackage{verbatim}

\usepackage{tabto}

\graphicspath{ {imgs/} }

\newcommand{\N}{\mathbb{N}}
\newcommand{\Z}{\mathbb{Z}}

% for the theorems, proofs, definitions
%\theoremstyle{plain}
%\newtheorem*{prop}{Afirmação}

\newcommand{\indbase}{\textbf{\textit{\\Base: }}}
\newcommand{\indhipo}{\textbf{\textit{\\Hipótese de Indução: }}}
\newcommand{\indstep}{\textbf{\textit{\\Passo: }}}

\renewcommand\qedsymbol{$\blacksquare$}

\newcommand\floor[1]{\big\lfloor#1\big\rfloor}
\newcommand\ceil[1]{\big\lceil#1\big\rceil}
\newcommand\bgfrac[2]{\bigg(\dfrac{#1}{#2}\bigg)}


\begin{document}

\title{MAC 4722 - Linguagens, Autômatos e Computabilidade}
\author{Rodrigo Augusto Dias Faria\\
Departamento de Ciência da Computação - IME/USP}

\maketitle

\begin{center}
\textbf{\large{Lista 1}}
\end{center}

\noindent \textbf{Sipser 0.11} Encontre o erro na seguinte prova de todos os cavalos são da mesma cor.


\textsc{Afirmação:} Em qualquer conjunto de $h$ cavalos, todos os cavalos são da mesma cor.

\begin{proof} Por indução sobre $h$.

\indbase Para $h = 1$. Em qualquer conjunto contendo somente um cavalo, todos os cavalos claramente têm a mesma cor.

\indstep Para $k \geq 1$ assuma que a afirmação seja verdadeira para $h = k$ e prove que ela é verdadeira para $h = k + 1$. Tome qualquer conjunto $H$ de $k + 1$ cavalos. Mostramos que todos os cavalos nesse conjunto são da mesma cor. Remova um cavalo desse conjunto para obter o conjunto $H_1$ com apenas $k$ cavalos. Pela hipótese da indução, todos os cavalos em $H_1$ são da mesma cor.

Agora recoloque o cavalo removido e remova um diferente para obter o conjunto $H_2$. Pelo mesmo argumento, todos os cavalos em $H_2$ são da mesma cor. Por conseguinte, todos os cavalos em $H$ têm que ser da mesma cor, e a prova está
completa.
\end{proof}

\textbf{Resposta:} De fato, a prova por indução está devidamente elaborada, ou seja, apresenta a base, hipótese e passo da indução e, no desenrolar da demonstração, a hipótese é aplicada para provar a afirmação dada.

Ocorre que o argumento é válido para quase todo $h$, porém falha quando $h = 2$, ou seja, quando $k = 1$, vejamos.

Seja $H = \{h_1, h_2\}$. Se removermos $h_2$, teremos um novo conjunto com um único cavalo $H_1 = \{h_1\}$ e, obviamente, todos os cavalos em $H_1$ são da mesma cor. O mesmo vale se retirarmos $h_1$ de $H$, o que nos dá outro conjunto $H_2 = \{h_2\}$. Mas isso é insuficiente para concluir que todos os cavalos em $H$ têm a mesma cor, uma vez que os cavalos em $H_1$ e $H_2$ podem ter cores diferentes.\\[6pt]
\noindent L1.1 Dado um grafo $G$ sem laços nem arestas múltiplas, dizemos que $G$ é uma árvore se qualquer par de vértices distintos é interligado por um único caminho que não repete vértices. Demonstre que toda Árvore $G$ com pelo menos $n \geq 1$ vértices possui exatamente $n - 1$ arestas. Sugestão: aplique uma indução em $n$; ou suponha por absurdo a existência de um contraexemplo com quantidade mínima de arestas.

\begin{proof}
Prova por indução em $n$.\\

Sejam $V$ e $E$ o conjunto de vértices e arestas do grafo $G$, respectivamente.\\

\textbf{Base:} para $n = 1$, temos:
\begin{align*}
|E[G]| = 0 &= n - 1 \\
      &= 1 - 1 \\
      &= 0
\end{align*}

\textbf{Hipótese de Indução:} assuma que a afirmação é verdadeira para qualquer árvore com um número de vértices menor do que $n$.\\

\textbf{Passo:} vamos considerar uma árvore $T$ com $n$ vértices. Seja $e$ uma aresta que conecta dois vértices $u$ e $v$ em $T$. Por definição de árvore, o único caminho entre $u$ e $v$ é a aresta $e$. Vamos remover $e$. Logo, teremos dois novos componentes (conexos e acíclicos), que também são árvores, $T'$ e $T''$, onde:
\begin{align*}
|V[T']|  &= n'  \text{,}\\
|V[T'']| &= n'' \quad \text{e} \\
n' + n'' &= n
\end{align*}

Ambas $T'$ e $T''$ têm menos vértices que $T$, logo:
\begin{align*}
|E[T']|  &= n' - 1   \quad (\text{por indução}) \\
|E[T'']| &= n'' - 1 \quad (\text{por indução})
\end{align*}

Não é difícil perceber que $T' \cup T''$ possui $(n' - 1) + (n'' - 1) = n -2$ arestas e, portanto, se adicionarmos $e$ de volta, teremos $n - 1$ arestas.

Como queríamos demonstrar!

\end{proof}

\noindent \textbf{L1.02} Uma palavra (cadeia) $x$ é dita uma potência de uma palavra $z$ se existir um inteiro $n \geq 0$ tal que $z^n = x$. Duas palavras $x$ e $y$ comutam entre si se $xy = yx$. Prove que duas palavras dadas $x$ e $y$ comutam se e somente se existir uma palavra $z$ da qual $x$ e $y$ são potências. Sugestão: no caso mais difícil, aplique uma indução na soma dos comprimentos de $x$ e $y$.

\textbf{Resposta:} $\Leftarrow$ \textsc{Afirmação:} Se $x = z^i$, $y = z^j$, então $xy = yx \quad \forall i, j \in \N$

Podemos provar essa afirmação da seguinte forma:
\begin{align*}
    xy = z^iz^j = z^{i+j} \\
    yx = z^jz^i = z^{j+i} \\
    \therefore xy = yx
\end{align*}

$\Rightarrow$ \textsc{Afirmação:} Se $xy = yx$, então $\exists z | z^i = x, z^j = y \quad \forall i, j \in \N$

\begin{proof} Prova por indução no tamanho da palavra.

\indbase Para $|x| = 0$ ou $|y| = 0$, temos:

Por definição de $\epsilon$ sabemos que $|\epsilon| = 0$.

No caso em que $|x| = 0$, temos que $x = \epsilon = z^0$. O mesmo vale no caso em que $|y| = 0$.

Se $|x| = |y|$, como sabemos que $xy = yx$, então podemos concluir que $x = y$, pois cada carácter da concatenação do lado esquerdo é exatamente igual ao que está na mesma posição do lado direito.

\indhypo Vamos assumir que a afirmação vale para $|x| < |y|$.

\indstep Neste caso, como sabemos que as cadeias comutam, podemos escrever $y = xv$ onde $v \in z^n \quad \forall n \in \N$. Em outras palavras, $x$ é um prefixo de $y$.

Logo:
\begin{align*}
    xy   &= yx  \\
    xxv  &= xvx \\
    xv   &= vx  \\
    |xv| &< |xy| \quad (\text{por indução})
\end{align*}
\end{proof}

\end{document}
