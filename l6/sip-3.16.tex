
\noindent \textbf{L6.4 (Sipser 3.16)} Mostre que a coleção de linguagens Turing-reconhecíveis é fechada sob a operação de união, concatenação, estrela e intersecção.\\[6pt]
\textbf{Resposta: } Sejam $L_1$ e $L_2$ linguagens Turing-reconhecíveis e $M_1$ e $M_2$ máquinas de Turing que reconhecem $L_1$ e $L_2$, respectivamente. Vamos mostrar que a classe de linguagens Turing-reconhecíveis é fechada sob as seguintes operações:
\begin{enumerate}[label={\textbf{\alph*.}}]
    \item união\\[3pt]
    %%%% New answer
    Vamos construir uma máquina de Turing $M$ capaz de reconhecer a união $L_1 \cup L_2$ da seguinte forma:\\[3pt]
    $M =$ “Sobre a cadeia de entrada $w$:
    \begin{enumerate}[label={\textbf{\arabic*.}}, leftmargin=1in]
        \item Repita o seguinte para $i = 1, 2, 3, \ldots$
    
        \item Rode as máquinas $M_1$ e $M_2$ sobre $w$ por $i$ passos. Se $M_1$ ou $M_2$ aceita $w$, \textit{aceite}. Se $M_1$ e $M_2$ param e rejeitam $w$, então \textit{rejeite}.”

    \end{enumerate}
    
    A máquina construída reconhece cadeias de $L_1 \cup L_2$, pois se $M_1$ ou $M_2$ aceitam $w$, em algum momento a máquina $M$ aceitará $w$, já que a máquina vai, passo a passo, tentando reconhecer $w$ em $M_1$ e $M_2$ simultaneamente. $M$ poderá entrar em \textit{loop} se $M_1$ ou $M_2$ entrar em \textit{loop} e ambas rejeitarem $w$.
    
    Referência na resposta do livro \cite{sipser:2006}.
    
    \item concatenação\\[3pt]
    %%%% New answer
    Vamos construir uma máquina de Turing $M$ capaz de reconhecer a concatenação $L_1L_2$.
    Seja $w$ uma cadeia de comprimento $n$, tal que pode ser segmentada em duas partes. Seja $p_i = x_iy_{n-i}$ cada uma dessas possíveis partições, onde $i = 0, 1, \ldots, n$ representa a quantidade de caracteres de $w$ (a partir do início) no primeiro segmento da partição e $n - i$ representa a quantidade de caracteres de $w$ (a partir de $i + 1$) no segundo segmento da partição.
    Vamos inserir na fita da máquina $M$ cada uma das $p_i$ partições separadas por um símbolo, digamos $\texttt{\#}$, da forma $\texttt{\#}p_0\texttt{\#}p_1\texttt{\#}\ldots p_n\texttt{\#}$, e cada segmento da partição $p_i$ separado por um outro caractere, digamos $\beta$, da forma $x_i\beta y_{n-i}$, ou seja, a fita ficaria da forma $\texttt{\#}x_0\beta y_n\texttt{\#}x_1\beta y_{n-1}\texttt{\#}\ldots \texttt{\#}x_n\beta y_0$. O que faremos é rodar a máquina $M_1$, simultaneamente, em todos os segmentos $x_i$ de cada partição $p_i$ e a máquina $M_2$ em todos os segmentos $y_{n-i}$ de cada partição $p_i$.
    Se $M_1$ e $M_2$ aceitar alguma partição $p_i$, onde $M_1$ aceita o primeiro segmento e $M_2$ aceita o segundo, então $M$ \textit{aceita} $w$. Se para todo $i$, $M_1$ e $M_2$ rejeitar $p_i$, onde $M_1$ rejeita o primeiro segmento ou $M_2$ rejeita o segundo, então $M$ \textit{rejeita} $w$. A máquina $M$ pode entrar em \textit{loop} se $w \notin L_1L_2$ e $M_1$ ou $M_2$ entrar em \textit{loop} na tentativa de reconhecimento das partições.
    
    \item estrela\\[3pt]
    %%%% New answer
    Vamos construir uma máquina de Turing $M$ capaz de reconhecer a operação estrela $L_1^*$.
    Seja $w$ uma cadeia de comprimento $n$, tal que pode ser particionada em $m$ partes. Seja $p_i = w'_{i1}w'_{i2} \ldots w'_{im}$ , para $i = 0, 1, \ldots$ cada uma dessas possíveis partições.
    Vamos inserir na fita da máquina $M$ cada uma das $p_i$ partições separadas por um símbolo, digamos $\texttt{\#}$, da forma $\texttt{\#}p_0\texttt{\#}p_1\texttt{\#} \ldots p_n\texttt{\#}$, e cada parte da partição $p_i$ separada por um outro caractere, digamos $\beta$, da forma $w'_{i1}\beta w'_{i2} \ldots$, ou seja, a fita ficaria da forma $\texttt{\#}w'_{01}\beta w'_{02}\beta \ldots w'_{0m}\texttt{\#}w'_{11}\beta w'_{12}\beta \ldots \texttt{\#}$.
    O que faremos é rodar a máquina $M_1$ simultaneamente em todas as partes $w'_{i1}w'_{i2}\ldots w'_{im}$ de cada partição $p_i$. Se $M_1$ aceitar alguma partição $p_i$, onde $M_1$ aceita cada uma das partes de $p_i$, então $M$ \textit{aceita} $w$. Se para todo $i$, $M_1$ rejeitar $p_i$, então $M$ \textit{rejeita} $w$. A máquina $M$ pode entrar em \textit{loop} se $w \notin L_1^*$ e $M_1$ entrar em \textit{loop} na tentativa de reconhecimento das partições.
    
    \item intersecção\\[3pt]
    %%%% New answer
    Vamos construir uma máquina de Turing $M$ capaz de reconhecer a intersecção $L_1 \cap L_2$ da seguinte forma:\\[3pt]
    $M =$ “Sobre a cadeia de entrada $w$:
    \begin{enumerate}[label={\textbf{\arabic*.}}, leftmargin=1in]
        \item Repita o seguinte para $i = 1, 2, 3, \ldots$
    
        \item Rode as máquinas $M_1$ e $M_2$ sobre $w$ por $i$ passos. Se $M_1$ e $M_2$ aceitam $w$, \textit{aceite}. Se $M_1$ ou $M_2$ para e rejeita $w$, então \textit{rejeite}.”
    \end{enumerate}
    
    A máquina construída aceita cadeias de $L_1 \cap L_2$, pois se $w \in L_1 \cap L_2$, significa que $\exists w \ |\ w \in L_1$ e $w \in L_2$. Como a máquina testa $w$ em $M_1$ e $M_2$, e só aceita caso $w$ seja aceita por ambas, temos que a máquina $M$ reconhece as cadeias de $L_1 \cap L_2$. $M$ poderá entrar em \textit{loop} se $M_1$ ou $M_2$ entrar em \textit{loop} e rejeitar $w$.
    
    Note que, no caso do fechamento sob concatenação e estrela, como existe apenas uma quantidade finita de fatorações de $w$, uma máquina de Turing pode tentar todas as possibilidades em um número finito de passos.
\end{enumerate}