% --------------------------------------------------------------
% This is all preamble stuff that you don't have to worry about.
% Head down to where it says "Start here"
% --------------------------------------------------------------

\documentclass[12pt]{article}

\usepackage[latin1,utf8]{inputenc}
\usepackage[brazil]{babel}

\usepackage[margin=1in]{geometry}
\usepackage{amsmath,amsthm,amssymb}
\usepackage{clrscode3e} % for the pseudocode
\usepackage{graphicx}
\usepackage{float}
\usepackage{framed}
\usepackage{caption}
% for tables
\usepackage{booktabs}
\usepackage[table,xcdraw]{xcolor}
% for the graphs %
\usepackage{tikz}
\usetikzlibrary{trees}
\usetikzlibrary{shapes,snakes}
\usepackage{verbatim}

\usepackage{tabto}

\graphicspath{ {imgs/} }

\newcommand{\N}{\mathbb{N}}
\newcommand{\Z}{\mathbb{Z}}

\newenvironment{theorem}[2][Theorem]{\begin{trivlist}
\item[\hskip \labelsep {\bfseries #1}\hskip \labelsep {\bfseries #2.}]}{\end{trivlist}}
\newenvironment{lemma}[2][Lemma]{\begin{trivlist}
\item[\hskip \labelsep {\bfseries #1}\hskip \labelsep {\bfseries #2.}]}{\end{trivlist}}
\newenvironment{exercise}[2][Exercise]{\begin{trivlist}
\item[\hskip \labelsep {\bfseries #1}\hskip \labelsep {\bfseries #2.}]}{\end{trivlist}}
\newenvironment{reflection}[2][Reflection]{\begin{trivlist}
\item[\hskip \labelsep {\bfseries #1}\hskip \labelsep {\bfseries #2.}]}{\end{trivlist}}
\newenvironment{proposition}[2][Proposition]{\begin{trivlist}
\item[\hskip \labelsep {\bfseries #1}\hskip \labelsep {\bfseries #2.}]}{\end{trivlist}}
\newenvironment{corollary}[2][Corollary]{\begin{trivlist}
\item[\hskip \labelsep {\bfseries #1}\hskip \labelsep {\bfseries #2.}]}{\end{trivlist}}

\newcommand\floor[1]{\big\lfloor#1\big\rfloor}
\newcommand\ceil[1]{\big\lceil#1\big\rceil}
\newcommand\bgfrac[2]{\bigg(\dfrac{#1}{#2}\bigg)}


\begin{document}

\title{MAC 4722 - Linguagens, Autômatos e Computabilidade}
\author{Rodrigo Augusto Dias Faria\\
Departamento de Ciência da Computação - IME/USP}

\maketitle

\begin{center}
\textbf{\large{Lista 1}}
\end{center}
\noindent L1.1 Dado um grafo $G$ sem laços nem arestas múltiplas, dizemos que $G$ é uma árvore se qualquer par de vértices distintos é interligado por um único caminho que não repete vértices. Demonstre que toda Árvore $G$ com pelo menos $n \geq 1$ vértices possui exatamente $n - 1$ arestas. Sugestão: aplique uma indução em $n$; ou suponha por absurdo a existência de um contraexemplo com quantidade mínima de arestas.

\begin{proof}
Prova por indução em $n$.\\

Sejam $V$ e $E$ o conjunto de vértices e arestas do grafo $G$, respectivamente.\\

\textbf{Base:} para $n = 1$, temos:
\begin{align*}
|E[G]| = 0 &= n - 1 \\
      &= 1 - 1 \\
      &= 0
\end{align*}

\textbf{Hipótese de Indução:} assuma que a afirmação é verdadeira para qualquer árvore com um número de vértices menor do que $n$.\\

\textbf{Passo:} vamos considerar uma árvore $T$ com $n$ vértices. Seja $e$ uma aresta que conecta dois vértices $u$ e $v$ em $T$. Por definição de árvore, o único caminho entre $u$ e $v$ é a aresta $e$. Vamos remover $e$. Logo, teremos dois novos componentes (conexos e acíclicos), que também são árvores, $T'$ e $T''$, onde:
\begin{align*}
|V[T']|  &= n'  \text{,}\\
|V[T'']| &= n'' \quad \text{e} \\
n' + n'' &= n
\end{align*}

Ambas $T'$ e $T''$ têm menos vértices que $T$, logo:
\begin{align*}
|E[T']|  &= n' - 1   \quad (\text{por indução}) \\
|E[T'']| &= n'' - 1 \quad (\text{por indução})
\end{align*}

Não é difícil perceber que $T' \cup T''$ possui $(n' - 1) + (n'' - 1) = n -2$ arestas e, portanto, se adicionarmos $e$ de volta, teremos $n - 1$ arestas.

Como queríamos demonstrar!

\end{proof}

\end{document}
