
\noindent \textbf{L4.6 Sipser(2.27)}
\begin{enumerate}[label={\textbf{\alph*.}}]
    
\item Mostre que $G$ é uma gramática ambígua.\\[3pt]
\textbf{Resposta: } Podemos mostrar que $G$ é ambígua quando em uma derivação ela produz duas árvores sintáticas distintas para uma mesma cadeia. Seja $w = $ \textbf{if condition then if condition then a := 1 else a := 1}.

\begin{figure}[H]
    \centering

    \begin{tikzpicture}
    \tikzset{frontier/.style={distance from root=200pt,sibling distance=22pt}}
    \Tree [.STMT 
            [.{IF-THEN} \textbf{if} \textbf{condition} \textbf{then}
                [.STMT  [.{IF-THEN-ELSE}
                            \textbf{if} \textbf{condition} \textbf{then}
                            [.STMT [.ASSIGN {\textbf{a := 1}} ] ]
                            \textbf{else}
                            [.STMT [.ASSIGN {\textbf{a := 1}} ] ]
                        ]
                ]
            ]
          ]
    \end{tikzpicture}
\end{figure}

\begin{figure}[H]
    \centering
    
    \begin{tikzpicture}
    \tikzset{frontier/.style={distance from root=200pt,sibling distance=22pt}}
    \Tree [.STMT 
            [.{IF-THEN-ELSE} \textbf{if} \textbf{condition} \textbf{then}
                [.STMT  [.{IF-THEN}
                            \textbf{if} \textbf{condition} \textbf{then}
                            [.STMT [.ASSIGN {\textbf{a := 1}} ] ] 
                        ]
                ]
                \textbf{else}
                [.STMT [.ASSIGN {\textbf{a := 1}} ] ]
            ]
          ]
    \end{tikzpicture}
\end{figure}

\end{enumerate}