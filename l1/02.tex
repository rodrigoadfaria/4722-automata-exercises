
\noindent \textbf{L1.02} Uma palavra (cadeia) $x$ é dita uma potência de uma palavra $z$ se existir um inteiro $n \geq 0$ tal que $z^n = x$. Duas palavras $x$ e $y$ comutam entre si se $xy = yx$. Prove que duas palavras dadas $x$ e $y$ comutam se e somente se existir uma palavra $z$ da qual $x$ e $y$ são potências. Sugestão: no caso mais difícil, aplique uma indução na soma dos comprimentos de $x$ e $y$.

\textbf{Resposta:} $\Leftarrow$ \textsc{Afirmação:} Se $x = z^i$, $y = z^j$, então $xy = yx \quad \forall i, j \in \N$

Podemos provar essa afirmação da seguinte forma:
\begin{align*}
    xy = z^iz^j = z^{i+j} \\
    yx = z^jz^i = z^{j+i} \\
    \therefore xy = yx
\end{align*}

$\Rightarrow$ \textsc{Afirmação:} Se $xy = yx$, então $\exists z | z^i = x, z^j = y \quad \forall i, j \in \N$

\begin{proof} Prova por indução no tamanho da palavra.

\indbase Para $|x| = 0$ ou $|y| = 0$, temos:

Por definição de $\epsilon$ sabemos que $|\epsilon| = 0$.

No caso em que $|x| = 0$, temos que $x = \epsilon = z^0$. O mesmo vale no caso em que $|y| = 0$.

Se $|x| = |y|$, como sabemos que $xy = yx$, então podemos concluir que $x = y$, pois cada carácter da concatenação do lado esquerdo é exatamente igual ao que está na mesma posição do lado direito.

\indhypo Vamos assumir que a afirmação vale para $|x| < |y|$.

\indstep Neste caso, como sabemos que as cadeias comutam, podemos escrever $y = xv$ onde $v \in z^n \quad \forall n \in \N$. Em outras palavras, $x$ é um prefixo de $y$.

Logo:
\begin{align*}
    xy   &= yx  \\
    xxv  &= xvx \\
    xv   &= vx  \\
    |xv| &< |xy| \quad (\text{por indução})
\end{align*}
\end{proof}