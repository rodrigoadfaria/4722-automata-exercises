
\noindent \textbf{L4.4 (Sipser 2.14)} Converta a seguinte GLC numa GLC equivalente na forma normal de Chomsky,
usando o procedimento dado no Teorema 2.9.
\begin{align*}
    A &\rightarrow BAB \ |\ B \ |\ \epsilon \\
    B &\rightarrow 00 \ |\ \epsilon
\end{align*}
\textbf{Resposta: } Seguem os passos de acordo com o teorema.
\begin{enumerate}
    \item Nova variável inicial
    \begin{align*}
        S_0 &\rightarrow A \\
        A &\rightarrow BAB \ |\ B \ |\ \epsilon \\
        B &\rightarrow 00 \ |\ \epsilon
    \end{align*}

    \item Removendo a regra $A \rightarrow \epsilon$
    \begin{align*}
        S_0 &\rightarrow A \ |\ \epsilon \\
        A &\rightarrow BAB \ |\ B \ |\ BB \\
        B &\rightarrow 00 \ |\ \epsilon
    \end{align*}

    \item Removendo a regra $B \rightarrow \epsilon$
    \begin{align*}
        S_0 &\rightarrow A \ |\ \epsilon \\
        A &\rightarrow BAB \ |\ B \ |\ BB \ |\ AB \ |\ BA \\
        B &\rightarrow 00
    \end{align*}

    \item Removendo a regra unitária $A \rightarrow B$
    \begin{align*}
        S_0 &\rightarrow A \ |\ \epsilon \\
        A &\rightarrow BAB \ |\ 00 \ |\ BB \ |\ AB \ |\ BA \\
        B &\rightarrow 00
    \end{align*}

    \item Removendo a regra unitária $S_0 \rightarrow a$
    \begin{align*}
        S_0 &\rightarrow BAB \ |\ 00 \ |\ BB \ |\ AB \ |\ BA  \ |\ \epsilon \\
        A &\rightarrow BAB \ |\ 00 \ |\ BB \ |\ AB \ |\ BA \\
        B &\rightarrow 00
    \end{align*}

    \item Simplificando, tomando $X \rightarrow AB$ e $Y \rightarrow 0$
    \begin{align*}
        S_0 &\rightarrow BX \ |\ YY \ |\ BB \ |\ AB \ |\ BA  \ |\ \epsilon \\
        A &\rightarrow BX \ |\ YY \ |\ BB \ |\ AB \ |\ BA \\
        B &\rightarrow YY \\
        X &\rightarrow AB \\
        Y &\rightarrow 0
    \end{align*}

\end{enumerate}