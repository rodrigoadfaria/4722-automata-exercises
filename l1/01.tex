
\noindent \textbf{L1.01} Dado um grafo $G$ sem laços nem arestas múltiplas, dizemos que $G$ é uma árvore se qualquer par de vértices distintos é interligado por um único caminho que não repete vértices. Demonstre que toda Árvore $G$ com pelo menos $n \geq 1$ vértices possui exatamente $n - 1$ arestas. Sugestão: aplique uma indução em $n$; ou suponha por absurdo a existência de um contraexemplo com quantidade mínima de arestas.

\textbf{Resposta:}
\begin{proof}
Prova por indução em $n$.\\

Sejam $V$ e $E$ o conjunto de vértices e arestas do grafo $G$, respectivamente.

\indbase Para $n = 1$, temos:
\begin{align*}
|E[G]| = 0 &= n - 1 \\
      &= 1 - 1 \\
      &= 0
\end{align*}

\indhipo Assuma que a afirmação é verdadeira para qualquer árvore com um número de vértices menor do que $n$.

\indstep Vamos considerar uma árvore $T$ com $n$ vértices. Seja $e$ uma aresta que conecta dois vértices $u$ e $v$ em $T$. Por definição de árvore, o único caminho entre $u$ e $v$ é a aresta $e$. Vamos remover $e$. Logo, teremos dois novos componentes (conexos e acíclicos), que também são árvores, $T'$ e $T''$, onde:
\begin{align*}
|V[T']|  &= n'  \text{,}\\
|V[T'']| &= n'' \quad \text{e} \\
n' + n'' &= n
\end{align*}

Ambas $T'$ e $T''$ têm menos vértices que $T$, logo:
\begin{align*}
|E[T']|  &= n' - 1   \quad (\text{por indução}) \\
|E[T'']| &= n'' - 1 \quad (\text{por indução})
\end{align*}

Não é difícil perceber que $T' \cup T''$ possui $(n' - 1) + (n'' - 1) = n -2$ arestas e, portanto, se adicionarmos $e$ de volta, teremos $n - 1$ arestas.

\end{proof}