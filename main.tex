% --------------------------------------------------------------
% This is all preamble stuff that you don't have to worry about.
% Head down to where it says "Start here"
% --------------------------------------------------------------

\documentclass[12pt]{article}

\usepackage[utf8]{inputenc}
\usepackage[brazil]{babel}
\usepackage{csquotes}

\usepackage[margin=1in]{geometry}
\usepackage{amsmath,amsthm,amssymb}
\usepackage{clrscode3e} % for the pseudocode
\usepackage{graphicx}
\usepackage[obeyspaces]{url}
\usepackage{float}
\usepackage{framed}
\usepackage{caption}
% for tables
\usepackage{booktabs}
\usepackage[table,xcdraw]{xcolor}
% for the graphs %
\usepackage{tikz}
\usetikzlibrary{trees}
\usetikzlibrary{shapes,decorations}
\usetikzlibrary{arrows,automata}
\usepackage{tikz-qtree}
\usepackage{verbatim}

\usepackage{tabto}
\usepackage{enumitem}

\usepackage[backend=bibtex,bibencoding=ascii,sorting=none]{biblatex}
\addbibresource{sources.bib}

\graphicspath{ {imgs/} }

\newcommand{\N}{\mathbb{N}}
\newcommand{\Z}{\mathbb{Z}}
\DeclareMathOperator{\deltaM}{\hat{\delta}_M}

% for the theorems, proofs, definitions
%\theoremstyle{plain}
%\newtheorem*{prop}{Afirmação}

\newcommand{\indbase}{\textbf{\textit{\\Base: }}}
\newcommand{\indhypo}{\textbf{\textit{\\Hipótese de Indução: }}}
\newcommand{\indstep}{\textbf{\textit{\\Passo: }}}

\renewcommand\qedsymbol{$\blacksquare$}
\newcommand*\rot{\rotatebox{90}}

\newcommand\floor[1]{\big\lfloor#1\big\rfloor}
\newcommand\ceil[1]{\big\lceil#1\big\rceil}
\newcommand\bgfrac[2]{\bigg(\dfrac{#1}{#2}\bigg)}


\newcommand\undermat[2]{% http://tex.stackexchange.com/a/102468/5764
  \makebox[0pt][l]{$\smash{\underbrace{\phantom{%
    \begin{matrix}#2\end{matrix}}}_{\text{$#1$}}}$}#2}
    
\begin{document}

\title{MAC 4722 - Linguagens, Autômatos e Computabilidade}
\author{Rodrigo Augusto Dias Faria - NUSP 9374992\\
Departamento de Ciência da Computação - IME/USP}

\maketitle

\begin{comment}
\begin{center}
\textbf{\large{Lista 1}}
\end{center}

\noindent \textbf{Sipser 0.11} Encontre o erro na seguinte prova de todos os cavalos são da mesma cor.


\textsc{Afirmação:} Em qualquer conjunto de $h$ cavalos, todos os cavalos são da mesma cor.

\begin{proof} Por indução sobre $h$.

\indbase Para $h = 1$. Em qualquer conjunto contendo somente um cavalo, todos os cavalos claramente têm a mesma cor.

\indstep Para $k \geq 1$ assuma que a afirmação seja verdadeira para $h = k$ e prove que ela é verdadeira para $h = k + 1$. Tome qualquer conjunto $H$ de $k + 1$ cavalos. Mostramos que todos os cavalos nesse conjunto são da mesma cor. Remova um cavalo desse conjunto para obter o conjunto $H_1$ com apenas $k$ cavalos. Pela hipótese da indução, todos os cavalos em $H_1$ são da mesma cor.

Agora recoloque o cavalo removido e remova um diferente para obter o conjunto $H_2$. Pelo mesmo argumento, todos os cavalos em $H_2$ são da mesma cor. Por conseguinte, todos os cavalos em $H$ têm que ser da mesma cor, e a prova está
completa.
\end{proof}

\textbf{Resposta:} De fato, a prova por indução está devidamente elaborada, ou seja, apresenta a base, hipótese e passo da indução e, no desenrolar da demonstração, a hipótese é aplicada para provar a afirmação dada.

Ocorre que o argumento é válido para quase todo $h$, porém falha quando $h = 2$, ou seja, quando $k = 1$, vejamos.

Seja $H = \{h_1, h_2\}$. Se removermos $h_2$, teremos um novo conjunto com um único cavalo $H_1 = \{h_1\}$ e, obviamente, todos os cavalos em $H_1$ são da mesma cor. O mesmo vale se retirarmos $h_1$ de $H$, o que nos dá outro conjunto $H_2 = \{h_2\}$. Mas isso é insuficiente para concluir que todos os cavalos em $H$ têm a mesma cor, uma vez que os cavalos em $H_1$ e $H_2$ podem ter cores diferentes.\\[6pt]
\noindent L1.1 Dado um grafo $G$ sem laços nem arestas múltiplas, dizemos que $G$ é uma árvore se qualquer par de vértices distintos é interligado por um único caminho que não repete vértices. Demonstre que toda Árvore $G$ com pelo menos $n \geq 1$ vértices possui exatamente $n - 1$ arestas. Sugestão: aplique uma indução em $n$; ou suponha por absurdo a existência de um contraexemplo com quantidade mínima de arestas.

\begin{proof}
Prova por indução em $n$.\\

Sejam $V$ e $E$ o conjunto de vértices e arestas do grafo $G$, respectivamente.\\

\textbf{Base:} para $n = 1$, temos:
\begin{align*}
|E[G]| = 0 &= n - 1 \\
      &= 1 - 1 \\
      &= 0
\end{align*}

\textbf{Hipótese de Indução:} assuma que a afirmação é verdadeira para qualquer árvore com um número de vértices menor do que $n$.\\

\textbf{Passo:} vamos considerar uma árvore $T$ com $n$ vértices. Seja $e$ uma aresta que conecta dois vértices $u$ e $v$ em $T$. Por definição de árvore, o único caminho entre $u$ e $v$ é a aresta $e$. Vamos remover $e$. Logo, teremos dois novos componentes (conexos e acíclicos), que também são árvores, $T'$ e $T''$, onde:
\begin{align*}
|V[T']|  &= n'  \text{,}\\
|V[T'']| &= n'' \quad \text{e} \\
n' + n'' &= n
\end{align*}

Ambas $T'$ e $T''$ têm menos vértices que $T$, logo:
\begin{align*}
|E[T']|  &= n' - 1   \quad (\text{por indução}) \\
|E[T'']| &= n'' - 1 \quad (\text{por indução})
\end{align*}

Não é difícil perceber que $T' \cup T''$ possui $(n' - 1) + (n'' - 1) = n -2$ arestas e, portanto, se adicionarmos $e$ de volta, teremos $n - 1$ arestas.

Como queríamos demonstrar!

\end{proof}

\noindent \textbf{L1.02} Uma palavra (cadeia) $x$ é dita uma potência de uma palavra $z$ se existir um inteiro $n \geq 0$ tal que $z^n = x$. Duas palavras $x$ e $y$ comutam entre si se $xy = yx$. Prove que duas palavras dadas $x$ e $y$ comutam se e somente se existir uma palavra $z$ da qual $x$ e $y$ são potências. Sugestão: no caso mais difícil, aplique uma indução na soma dos comprimentos de $x$ e $y$.

\textbf{Resposta:} $\Leftarrow$ \textsc{Afirmação:} Se $x = z^i$, $y = z^j$, então $xy = yx \quad \forall i, j \in \N$

Podemos provar essa afirmação da seguinte forma:
\begin{align*}
    xy = z^iz^j = z^{i+j} \\
    yx = z^jz^i = z^{j+i} \\
    \therefore xy = yx
\end{align*}

$\Rightarrow$ \textsc{Afirmação:} Se $xy = yx$, então $\exists z | z^i = x, z^j = y \quad \forall i, j \in \N$

\begin{proof} Prova por indução no tamanho da palavra.

\indbase Para $|x| = 0$ ou $|y| = 0$, temos:

Por definição de $\epsilon$ sabemos que $|\epsilon| = 0$.

No caso em que $|x| = 0$, temos que $x = \epsilon = z^0$. O mesmo vale no caso em que $|y| = 0$.

Se $|x| = |y|$, como sabemos que $xy = yx$, então podemos concluir que $x = y$, pois cada carácter da concatenação do lado esquerdo é exatamente igual ao que está na mesma posição do lado direito.

\indhypo Vamos assumir que a afirmação vale para $|x| < |y|$.

\indstep Neste caso, como sabemos que as cadeias comutam, podemos escrever $y = xv$ onde $v \in z^n \quad \forall n \in \N$. Em outras palavras, $x$ é um prefixo de $y$.

Logo:
\begin{align*}
    xy   &= yx  \\
    xxv  &= xvx \\
    xv   &= vx  \\
    |xv| &< |xy| \quad (\text{por indução})
\end{align*}
\end{proof}

\begin{center}
\textbf{\large{Lista 2}}
\end{center}

\noindent \textbf{L2.1 (Sipser 1.16)} Resolva o exercício 1.16.\\[3pt]
\noindent\textbf{a) Resposta:} Vamos chamar de $M$ o AFN dado na questão. Seja $M_d = \{Q', \Sigma, \delta', q_{0'}, F'\}$ o AFD equivalente à $M$.

\noindent\textit{Estados de $M_d$:} $Q' = \{\{\}, \{1\}, \{2\}, \{1,2\} \}$.

\noindent\textit{Estado inicial:} $q_{0'} = E(\{1\}) = \{1\}$. É o conjunto de estados que são atingíveis a partir de $\{1\}$ viajando por setas $\epsilon$, mais o próprio $\{1\}$.

\noindent\textit{Estados de aceitação:} $F' = \{\{1\}, \{1,2\}\}$. Aqueles que contêm um estado de aceitação de $M$.

\noindent\textit{Função de transição:} $\delta' = $
\begin{table}[!h]
\centering
\rot{\hspace{5 mm}\llap{Estados}}
\begin{tabular}{l|l|l}
         & a         & b        \\ \hline
\{\}     & \{\}      & \{\}     \\
\{1\}    & \{1,2\}   & \{2\}    \\
\{2\}    & \{\}      & \{1\}    \\
\{1,2\}  & \{1,2\}   & \{1,2\}
\end{tabular}
\caption{Função de transição de $M_d$.}\vspace*{0.2cm}
\label{tbl:sip1.16a}
\end{table}

\begin{figure}[!h]
\centering
\begin{tikzpicture}[scale=0.2]
\tikzstyle{every node}+=[inner sep=0pt]
\draw [black] (17.6,-19.6) circle (3);
\draw (17.6,-19.6) node {$\{1\}$};
\draw [black] (17.6,-19.6) circle (2.4);
\draw [black] (47.5,-36) circle (3);
\draw (47.5,-36) node {$\{\}$};
\draw [black] (17.6,-36) circle (3);
\draw (17.6,-36) node {$\{2\}$};
\draw [black] (47.5,-19.6) circle (3);
\draw (47.5,-19.6) node {$\{1,2\}$};
\draw [black] (47.5,-19.6) circle (2.4);
\draw [black] (19.342,-22.033) arc (28.49627:-28.49627:12.088);
\fill [black] (19.34,-33.57) -- (20.16,-33.1) -- (19.28,-32.63);
\draw (21.31,-27.8) node [right] {$b$};
\draw [black] (50.18,-34.677) arc (144:-144:2.25);
\draw (54.75,-36) node [right] {$a,\mbox{ }b$};
\fill [black] (50.18,-37.32) -- (50.53,-38.2) -- (51.12,-37.39);
\draw [black] (20.6,-36) -- (44.5,-36);
\fill [black] (44.5,-36) -- (43.7,-35.5) -- (43.7,-36.5);
\draw (32.55,-35.5) node [above] {$a$};
\draw [black] (15.808,-33.604) arc (-150.49236:-209.50764:11.784);
\fill [black] (15.81,-22) -- (14.98,-22.45) -- (15.85,-22.94);
\draw (13.78,-27.8) node [left] {$b$};
\draw [black] (50.18,-18.277) arc (144:-144:2.25);
\draw (54.75,-19.6) node [right] {$a,\mbox{ }b$};
\fill [black] (50.18,-20.92) -- (50.53,-21.8) -- (51.12,-20.99);
\draw [black] (20.6,-19.6) -- (44.5,-19.6);
\fill [black] (44.5,-19.6) -- (43.7,-19.1) -- (43.7,-20.1);
\draw (32.55,-20.1) node [below] {$a$};
\draw [black] (10.5,-19.6) -- (14.6,-19.6);
\fill [black] (14.6,-19.6) -- (13.8,-19.1) -- (13.8,-20.1);
\end{tikzpicture}
\caption{Diagrama de estados para o AFD $M_d$.}
\label{fig:sip1.16a}
\end{figure}

% item b) %
\noindent\textbf{b) Resposta:} Vamos chamar de $N$ o AFN dado na questão. Seja $N_d = \{Q', \Sigma, \delta', q_{0'}, F'\}$ o AFD equivalente à $N$.

\noindent\textit{Estados de $N_d$:} $Q' = \{\{\}, \{1\}, \{2\}, \{3\}, \{1,2\}, \{1,3\}, \{2,3\}, \{1,2,3\} \}$.

\noindent\textit{Estado inicial:} $q_{0'} = E(\{1\}) = \{1,2\}$.

\noindent\textit{Estados de aceitação:} $F' = \{\{2\}, \{1,2\}, \{2,3\}, \{1,2,3\} \}$. Aqueles que contêm um estado de aceitação de $N$.

\noindent\textit{Função de transição:} $\delta' = $
\begin{table}[!h]
\centering
\rot{\hspace{5 mm}\llap{Estados}}
\begin{tabular}{l|l|l}
            & a             & b         \\ \hline
\{\}        & \{\}          & \{\}      \\
\{1\}       & \{3\}         & \{\}      \\
\{2\}       & \{1,2\}       & \{\}      \\
\{3\}       & \{2\}         & \{2,3\}   \\
\{1,2\}     & \{1,2,3\}     & \{\}      \\
\{1,3\}     & \{2,3\}       & \{2,3\}   \\
\{2,3\}     & \{1,2\}       & \{2,3\}   \\
\{1,2,3\}   & \{1,2,3\}     & \{2,3\}
\end{tabular}
\caption{Função de transição de $N_d$.}\vspace*{0.2cm}
\label{tbl:sip1.16b}
\end{table}

\begin{figure}[!h]
\centering
\begin{tikzpicture}[scale=0.2]
\tikzstyle{every node}+=[inner sep=0pt]
\draw [black] (17.6,-19.6) circle (3);
\draw (17.6,-19.6) node {$\{1,2\}$};
\draw [black] (17.6,-19.6) circle (2.4);
\draw [black] (47.5,-36) circle (3);
\draw (47.5,-36) node {${2,3}$};
\draw [black] (47.5,-36) circle (2.4);
\draw [black] (17.6,-36) circle (3);
\draw (17.6,-36) node {$\{\}$};
\draw [black] (47.5,-19.6) circle (3);
\draw (47.5,-19.6) node {$\{1,2,3\}$};
\draw [black] (47.5,-19.6) circle (2.4);
\draw [black] (50.18,-34.677) arc (144:-144:2.25);
\draw (54.75,-36) node [right] {$b$};
\fill [black] (50.18,-37.32) -- (50.53,-38.2) -- (51.12,-37.39);
\draw [black] (50.18,-18.277) arc (144:-144:2.25);
\draw (54.75,-19.6) node [right] {$a$};
\fill [black] (50.18,-20.92) -- (50.53,-21.8) -- (51.12,-20.99);
\draw [black] (20.6,-19.6) -- (44.5,-19.6);
\fill [black] (44.5,-19.6) -- (43.7,-19.1) -- (43.7,-20.1);
\draw (32.55,-20.1) node [below] {$a$};
\draw [black] (10.5,-19.6) -- (14.6,-19.6);
\fill [black] (14.6,-19.6) -- (13.8,-19.1) -- (13.8,-20.1);
\draw [black] (17.6,-22.6) -- (17.6,-33);
\fill [black] (17.6,-33) -- (18.1,-32.2) -- (17.1,-32.2);
\draw (17.1,-27.8) node [left] {$b$};
\draw [black] (18.923,-38.68) arc (54:-234:2.25);
\draw (17.6,-43.25) node [below] {$a,\mbox{ }b$};
\fill [black] (16.28,-38.68) -- (15.4,-39.03) -- (16.21,-39.62);
\draw [black] (47.5,-22.6) -- (47.5,-33);
\fill [black] (47.5,-33) -- (48,-32.2) -- (47,-32.2);
\draw (47,-27.8) node [left] {$b$};
\draw [black] (44.87,-34.56) -- (20.23,-21.04);
\fill [black] (20.23,-21.04) -- (20.69,-21.87) -- (21.17,-20.99);
\draw (33.49,-27.3) node [above] {$a$};
\end{tikzpicture}
\caption{Diagrama de estados para o AFD $N_d$.}
\label{fig:sip1.16b}
\end{figure}

A figura \ref{fig:sip1.16b} é o AFD simplificado que mostra apenas os estados que são alcançáveis a partir do estado inicial $\{1,2\}$.\\[6pt]

\noindent \textbf{L2.2 (Sipser 1.6c)} Dê um DFA/AFD para $A = \{w | w$ possui $0101$ por subcadeia\}.

\begin{figure}[!h]
\centering
\begin{tikzpicture}[scale=0.2]
    \tikzstyle{every node}+=[inner sep=0pt]
    \draw [black] (7.9,-19.1) circle (3);
    \draw (7.9,-19.1) node {$q_0$};
    \draw [black] (24.5,-19.1) circle (3);
    \draw (24.5,-19.1) node {$q_1$};
    \draw [black] (39.9,-19.1) circle (3);
    \draw (39.9,-19.1) node {$q_2$};
    \draw [black] (55.5,-19.1) circle (3);
    \draw (55.5,-19.1) node {$q_3$};
    \draw [black] (74.1,-19.1) circle (3);
    \draw (74.1,-19.1) node {$q_4$};
    \draw [black] (74.1,-19.1) circle (2.4);
    \draw [black] (0.8,-19.1) -- (4.9,-19.1);
    \fill [black] (4.9,-19.1) -- (4.1,-18.6) -- (4.1,-19.6);
    \draw [black] (10.9,-19.1) -- (21.5,-19.1);
    \fill [black] (21.5,-19.1) -- (20.7,-18.6) -- (20.7,-19.6);
    \draw (16.2,-19.6) node [below] {$0$};
    \draw [black] (27.5,-19.1) -- (36.9,-19.1);
    \fill [black] (36.9,-19.1) -- (36.1,-18.6) -- (36.1,-19.6);
    \draw (32.2,-18.6) node [above] {$1$};
    \draw [black] (42.9,-19.1) -- (52.5,-19.1);
    \fill [black] (52.5,-19.1) -- (51.7,-18.6) -- (51.7,-19.6);
    \draw (47.7,-19.6) node [below] {$0$};
    \draw [black] (58.5,-19.1) -- (71.1,-19.1);
    \fill [black] (71.1,-19.1) -- (70.3,-18.6) -- (70.3,-19.6);
    \draw (64.8,-19.6) node [below] {$1$};
    \draw [black] (6.577,-16.42) arc (234:-54:2.25);
    \draw (7.9,-11.85) node [above] {$1$};
    \fill [black] (9.22,-16.42) -- (10.1,-16.07) -- (9.29,-15.48);
    \draw [black] (23.177,-16.42) arc (234:-54:2.25);
    \draw (24.5,-11.85) node [above] {$0$};
    \fill [black] (25.82,-16.42) -- (26.7,-16.07) -- (25.89,-15.48);
    \draw [black] (26.483,-16.853) arc (134.14834:45.85166:19.407);
    \fill [black] (26.48,-16.85) -- (27.41,-16.65) -- (26.71,-15.94);
    \draw (40,-10.87) node [above] {$0$};
    \draw [black] (37.829,-21.267) arc (-47.84822:-132.15178:20.755);
    \fill [black] (9.97,-21.27) -- (10.23,-22.17) -- (10.9,-21.43);
    \draw (23.9,-27.13) node [below] {$1$};
    \draw [black] (72.777,-16.42) arc (234:-54:2.25);
    \draw (74.1,-11.85) node [above] {$0,\mbox{ }1$};
    \fill [black] (75.42,-16.42) -- (76.3,-16.07) -- (75.49,-15.48);
\end{tikzpicture}
\caption{Diagrama de estados do AFD que reconhece $A$.}
\label{fig:sip1.6c}
\end{figure}


\noindent \textbf{L2.3} Dada uma linguagem $L$, seja $Pref(L) = \{x |$ existe palavra $y$ tal que $xy$ está em $L$\}, $Suf(L) = \{y |$ existe palavra $x$ tal que $xy$ está em $L$\}, $Fat(L) = \{y |$ existem palavras $x$ e $z$ tais que $xyz$ estão em $L$\}.\\
Demonstre que se $L$ é regular, então $Pref(L)$, $Suf(L)$ e $Fat(L)$ também o são. Sugestão: Observe que $Fat(L) = Suf(Pref(L))$.\\[3pt]

\noindent \textbf{L2.4} Complete a demonstração do teorema 1.25.\\[3pt]
\textbf{Resposta:} Vale lembrar, resumidamente, da construção dada na prova do teorema 1.25.

Suponha que $A_1$ e $A_2$ são linguagens reconhecidas por $M_1$ e $M_2$, respectivamente, onde $M_1 = (Q_1, \Sigma, \delta_1, q_1, F_1)$ e $M_2 = (Q_2, \Sigma, \delta_2, q_2, F_2)$.

Construa $M$ para reconhecer $A_1 \cup A_2$, onde $M = (Q, \Sigma, \delta, q_0, F)$.
\begin{enumerate}[label=\textbf{\arabic*}]
    \item $Q = Q_1 \times Q_2$.
    \item $\Sigma$, o alfabeto, é o mesmo em $M_1$ e $M_2$.
    \item $\delta = $ para cada $(r_1, r_2) \in Q$ e cada $a \in \Sigma$, faça $\delta((r_1, r_2), a) = (\delta_1(r_1, a), \delta_2(r_2, a))$.
    \item $q_0 = (q_1, q_2)$.
    \item $F = (F_1 \times Q_2) \cup (Q_1 \times F_2)$.
\end{enumerate}

\begin{proof}
Para demonstrar que $M$ reconhece $A_1 \cup A_2$, devemos dividir a prova em duas partes.

\textsc{Afirmação:} Toda palavra pertencente à linguagem reconhecida por esse autômato está presente em $A_1 \cup A_2$.

Tome uma palavra $w$ qualquer reconhecida pelo autômato $M$. Sabe-se que ao transitarmos através de $\delta$ por $M$, a partir do estado inicial $q_0$, existe um passeio $P$ no autômato $M$ que leva a um estado final. Pela construção de $M$, cada estado nesse passeio é rotulado por um par ordenado $(r_1, r_2)$, onde $r_1 \in M_1$ e $r_2 \in M_2$. Se tomarmos o passeio $P_1$ considerando de $P$ apenas as coordenadas $r_1$ do par ordenado, este é equivalente ao passeio dado pelas transições $\delta_1$ na tentativa de reconhecimento de $w$ em $M_1$. Analogamente, podemos tomar o passeio $P_2$, a partir de $P$, considerando apenas as coordenadas $r_2$, o que equivaleria à tentativa de reconhecimento da palavra $w$ em $M_2$. Pela construção de $M$, temos ainda que o estado final do passeio $P$ é rotulado por um par ordenado $(r_1, r_2)$, onde $r_1 \in F_1$ ou $r_2 \in F_2$. Dessa forma, ou $P_1$ ou $P_2$, ou ambos, terminam com um estado final, logo, $w$ pertence ou a $A_1$, ou a $A_2$, ou a ambas, o que é equivalente a dizer que $w$ pertence à $A_1 \cup A_2$.

\textsc{Afirmação:} Toda palavra pertencente à linguagem $A_1 \cup A_2$ é reconhecida pelo autômato construído.
Tomemos agora uma cadeia $w$ como sendo uma cadeia pertencente a $A_1 \cup A_2$, onde $|w| = m$. Logo, existe um passeio $P_1 = x_0, x_1, \ldots, x_m$ em $M_1$, tal que $x_0 = q_1$ construído a partir de $\delta_1$, ou um passeio $P_2 = z_0, z_1, \ldots, z_m$, construído a partir de $\delta_2$ em $M_2$, tal que $z_0 = q2$, e que $x_m$ ou $z_m$, ou ambos, são estados finais. Como o conjunto de estados $Q$ de $M$ foi construído através do produto cartesiano de $Q_1 \times Q_2$ e a função de transição $\delta((r_1, r_2), a) = (\delta_1(r_1, a), \delta_2(r_2, a))$, para cada par ordenado $(r_1, r_2) \in Q$ e cada $a \in \Sigma$, existe um caminho $P = (x_0, z_0), (x_1, z_1), \ldots, (x_m,z_m)$ em $M$, obtido a partir de $w$, e como $x_m$ ou $z_m$, ou ambos, são estados finais, $(x_m, z_m)$ também é um estado final e, portanto, $M$ reconhece a palavra $w$.
\end{proof}

\begin{center}
\textbf{\large{Lista 3}}
\end{center}

\noindent \textbf{L3.1 (Sipser 1.46)} Prove que as seguintes linguagens não são regulares.
\begin{enumerate}[label={\textbf{\alph*.}}]
    \item $L_a = \{0^n1^m0^n \ |\ m, n \geq 0\}$\\[3pt]
    \textbf{Resposta:} Vamos usar o lema do bombeamento para mostrar que $L_a$ não é regular. A prova é por contradição.
    
    Suponha o contrário, ou seja, que $L_a$ é regular. Seja $p$ o comprimento de bombeamento dado pelo lema do bombeamento. Seja $s$ a cadeia $s = 0^p1^{p+1}0^p$. Como $s \in L_a$ e $|s| \geq p$, o lema do bombeamento garante que $s$ pode ser dividida em três partes $s = xyz$, onde, para qualquer $i \geq 0$, $xy^iz \in L_a$. Vamos mostrar que isso é impossível.
    
    A condição 3 do lema do bombeamento diz que $|xy| \leq p$ e, por esta razão, $y$ contém apenas 0s. Vamos tomar a cadeia $s' = xyyz$, onde $y = 0^a$ e $a \geq 1$. Neste caso, teremos mais 0s no início da cadeia do que no fim e, portanto, $s' \notin L_a$, o que é uma contradição da condição 1 do lema do bombeamento.
    
    Portanto, podemos concluir que $L_a$ não é regular.
    
    \item $L_b = \{0^m1^n \ |\ m \neq n\}$\\[3pt]
    \textbf{Resposta:} Há a resposta no livro do Sipser.
    
    \item $L_c = \{w \ |\ w \in \{0, 1\}^*$ não é um palíndromo$\}$\\[3pt]
    \textbf{Resposta:} Tomei como base a questão anterior no livro \cite{sipser:2006}.
    
    Um palíndromo é uma cadeia que tem a mesma leitura da esquerda para a direita e vice-versa. Logo, 
    $L_c = \{w \ |\ w \in \{0, 1\}^* e\ w \neq w^R\}$.
    
    Vamos usar o lema do bombeamento para mostrar que $L_c$ não é regular. A prova é por contradição.
    
    Suponha o contrário, ou seja, que $L_c$ é regular. Seja $p$ o comprimento de bombeamento dado pelo lema do bombeamento. Seja a cadeia $s = 0^p10^{p+p!}$. Como $s \in L_c$ e $|s| \geq p$, o lema do bombeamento garante que $s$ pode ser dividida em três partes $s = xyz$ com $x = 0^a$, $y = 0^b$ e $z = 0^c10^{p+p!}$, onde, $b \geq 1$ e $a + b + c = p$. Vamos mostrar que isso é impossível.
    
    Seja a cadeia $s' = xy^{i+1}z$, onde $i = \frac{p!}{b}$. Então, temos que:
    \begin{align*}
        y^{i+1} = 0^{b^{(\frac{p!}{b})}} 0^b = 0^{b{(\frac{p!}{b})}} 0^b = 0^{p!}0^b = 0^{b + p!}
    \end{align*}
    Logo, $xyz = 0^a0^{b + p!}0^c10^{p + p!} = 0^{a + b + p! + c}10^{b + p!}$. Como $a + b + c = p$, temos que $xyz = 0^{p + p!}10^{p + p!}$ e, sendo assim, $xyz \notin L_c$.
    
    Portanto, podemos concluir que $L_c$ não é regular.
    
    \item $L_d = \{wtw \ |\ w, t \in \{0, 1\}^+\}$\\[3pt]
    \textbf{Resposta:} Vamos usar o lema do bombeamento para mostrar que $L_d$ não é regular. A prova é por contradição.
    
    Suponha o contrário, ou seja, que $L_d$ é regular. Seja $p$ o comprimento de bombeamento dado pelo lema do bombeamento. Seja a cadeia $s = 0^p10^p$, onde $p \geq 1$. Como $s \in L_d$ e $|s| \geq p$, o lema do bombeamento garante que $s$ pode ser dividida em três partes $s = xyz$, onde, para qualquer $i \geq 0$, $xy^iz \in L_d$. Vamos mostrar que isso é impossível.
    
    A condição 3 do lema do bombeamento diz que $|xy| \leq p$ e, por esta razão, $y$ contém apenas 0s. Vamos tomar a cadeia $s' = xyyz$. Neste caso, teremos um número maior de 0s no início da cadeia do que no fim e, portanto, $s' \notin L_d$, o que é uma contradição da condição 1 do lema do bombeamento.
    
    Portanto, podemos concluir que $L_d$ não é regular.
\end{enumerate}

\noindent \textbf{L3.2 (Sipser 1.49)} Reescrevi o enunciado substituindo $y$ por $w$ para não confundir com as propriedades do lema do bombeamento. Para ambas as questões o alfabeto é $\Sigma = \{0, 1\}$.

\begin{enumerate}[label={\textbf{\alph*.}}]
    \item Seja $B = \{1^kw \ |\ w\ \in \Sigma^*$ e $w$ contém pelo menos $k$ 1s, para $k \geq 1\}$.\\
    Mostre que $B$ é uma linguagem regular.
    
    \textbf{Resposta:} Se $B$ é uma linguagem regular, então existe um autômato finito que a reconhece.
    
    Podemos observar algumas cadeias que pertencem a $B$, tais como, $111|00111$, $1|10$, $1|11$, $1|01$ e outras que não pertencem a $B$, tais como, $0|0$, $0|1$, $1|00$, onde $|$ indica o final da subcadeia $1^k$. Percebemos, então, que as cadeias em $B$ não podem começar com 0 e, as que estão em $B$ devem, necessariamente, começar em 1, já que $k \geq 1$.
    
    Logo, podemos reescrever $B$ como $B = \{1x \ |\ x \in \Sigma^*$ e $x$ tem pelo menos um 1$\}$.
    
    Agora, vamos construir um AFD $M = (Q, \Sigma, \delta, s, F)$ que reconhece $B$:
    \begin{center}
    \begin{tikzpicture}[scale=0.2]
    \tikzstyle{every node}+=[inner sep=0pt]
    \draw [black] (17.6,-24.8) circle (3);
    \draw (17.6,-24.8) node {$s$};
    \draw [black] (32.1,-24.8) circle (3);
    \draw (32.1,-24.8) node {$q_1$};
    \draw [black] (46.4,-18.4) circle (3);
    \draw (46.4,-18.4) node {$q_3$};
    \draw [black] (46.4,-31.9) circle (3);
    \draw (46.4,-31.9) node {$q_2$};
    \draw [black] (46.4,-31.9) circle (2.4);
    \draw [black] (9.5,-24.8) -- (14.6,-24.8);
    \fill [black] (14.6,-24.8) -- (13.8,-24.3) -- (13.8,-25.3);
    \draw [black] (16.277,-22.12) arc (234:-54:2.25);
    \draw (17.6,-17.55) node [above] {$0$};
    \fill [black] (18.92,-22.12) -- (19.8,-21.77) -- (18.99,-21.18);
    \draw [black] (20.6,-24.8) -- (29.1,-24.8);
    \fill [black] (29.1,-24.8) -- (28.3,-24.3) -- (28.3,-25.3);
    \draw (24.85,-25.3) node [below] {$1$};
    \draw [black] (34.84,-23.57) -- (43.66,-19.63);
    \fill [black] (43.66,-19.63) -- (42.73,-19.5) -- (43.14,-20.41);
    \draw (40.38,-22.11) node [below] {$0$};
    \draw [black] (34.79,-26.13) -- (43.71,-30.57);
    \fill [black] (43.71,-30.57) -- (43.22,-29.76) -- (42.77,-30.66);
    \draw (38.11,-28.86) node [below] {$1$};
    \draw [black] (46.4,-21.4) -- (46.4,-28.9);
    \fill [black] (46.4,-28.9) -- (46.9,-28.1) -- (45.9,-28.1);
    \draw (45.9,-25.15) node [left] {$1$};
    \draw [black] (45.077,-15.72) arc (234:-54:2.25);
    \draw (46.4,-11.15) node [above] {$0$};
    \fill [black] (47.72,-15.72) -- (48.6,-15.37) -- (47.79,-14.78);
    \draw [black] (47.723,-34.58) arc (54:-234:2.25);
    \draw (46.4,-39.15) node [below] {$0,1$};
    \fill [black] (45.08,-34.58) -- (44.2,-34.93) -- (45.01,-35.52);
    \end{tikzpicture}
    \end{center}
    

    \item Seja $C = \{1^kw \ |\ w\ \in \Sigma^*$ e $w$ contém no máximo $k$ 1s, para $k \geq 1\}$.\\
    Mostre que $C$ nâo é uma linguagem regular.
    
    \textbf{Resposta:} Vamos usar o lema do bombeamento para mostrar que $C$ não é regular. A prova é por contradição.
    
    Suponha o contrário, ou seja, que $C$ é regular. Seja $p$ o comprimento de bombeamento dado pelo lema do bombeamento. Seja $s$ a cadeia $s = 1^p01^p$. Como $s \in C$ e $|s| \geq p$, o lema do bombeamento garante que $s$ pode ser dividida em três partes $s = xyz$ com $x = 1^a$, $y = 1^b$ e $z = 1^c01^p$, onde, $b \geq 1$ e $a + b + c = p$. Vamos mostrar que isso é impossível.
    
    Vamos tomar a cadeia $s' = xy^0z = 1^{a+c}01^p$. Como $b \geq 1$ e $a + b + c = p$, nós temos que $a + c < p$ e, sendo assim, $s' \notin C$, o que contradiz a condição 1 do lema do bombeamento.
    
    Portanto, podemos concluir que $C$ não é regular.
\end{enumerate}

\noindent \textbf{L3.3} Converter a expressão regular $0(0\cup1)^*01(0\cup1)^*1$ para AFN.\\[3pt]
\textbf{Resposta:} As figuras abaixo mostram passo a passo a construção do AFN que representa a expressão dada.

\begin{tikzpicture}[scale=0.2]
\tikzstyle{every node}+=[inner sep=0pt]
\node[draw,align=left,inner sep=3pt] at (0,0) {$0$};
\draw [black] (9.5,-6.1) circle (3);
\draw [black] (26.2,-6.1) circle (3);
\draw [black] (26.2,-6.1) circle (2.4);
\draw [black] (2.2,-6.1) -- (6.5,-6.1);
\fill [black] (6.5,-6.1) -- (5.7,-5.6) -- (5.7,-6.6);
\draw [black] (12.5,-6.1) -- (23.2,-6.1);
\fill [black] (23.2,-6.1) -- (22.4,-5.6) -- (22.4,-6.6);
\draw (17.85,-6.6) node [below] {$0$};
\end{tikzpicture}


\begin{tikzpicture}[scale=0.2]
\tikzstyle{every node}+=[inner sep=0pt]
\node[draw,align=left,inner sep=3pt] at (0,0) {$1$};
\draw [black] (9.5,-6.1) circle (3);
\draw [black] (26.2,-6.1) circle (3);
\draw [black] (26.2,-6.1) circle (2.4);
\draw [black] (2.2,-6.1) -- (6.5,-6.1);
\fill [black] (6.5,-6.1) -- (5.7,-5.6) -- (5.7,-6.6);
\draw [black] (12.5,-6.1) -- (23.2,-6.1);
\fill [black] (23.2,-6.1) -- (22.4,-5.6) -- (22.4,-6.6);
\draw (17.85,-6.6) node [below] {$1$};
\end{tikzpicture}


\begin{tikzpicture}[scale=0.2]
\tikzstyle{every node}+=[inner sep=0pt]
\node[draw,align=left,inner sep=3pt] at (0,0) {$0\cup1$};
\draw [black] (56.6,-6.1) circle (3);
\draw [black] (72.2,-6.1) circle (3);
\draw [black] (72.2,-6.1) circle (2.4);
\draw [black] (56.6,-16.1) circle (3);
\draw [black] (72.2,-16.1) circle (3);
\draw [black] (72.2,-16.1) circle (2.4);
\draw [black] (44.3,-11.2) circle (3);
\draw [black] (59.6,-6.1) -- (69.2,-6.1);
\fill [black] (69.2,-6.1) -- (68.4,-5.6) -- (68.4,-6.6);
\draw (64.4,-6.6) node [below] {$0$};
\draw [black] (47.07,-10.05) -- (53.83,-7.25);
\fill [black] (53.83,-7.25) -- (52.9,-7.09) -- (53.28,-8.02);
\draw (51.34,-9.16) node [below] {$\epsilon$};
\draw [black] (47.09,-12.31) -- (53.81,-14.99);
\fill [black] (53.81,-14.99) -- (53.25,-14.23) -- (52.88,-15.16);
\draw (49.57,-14.17) node [below] {$\epsilon$};
\draw [black] (59.6,-16.1) -- (69.2,-16.1);
\fill [black] (69.2,-16.1) -- (68.4,-15.6) -- (68.4,-16.6);
\draw (64.4,-16.6) node [below] {$1$};
\draw [black] (36.1,-11.2) -- (41.3,-11.2);
\fill [black] (41.3,-11.2) -- (40.5,-10.7) -- (40.5,-11.7);
\end{tikzpicture}


\begin{tikzpicture}[scale=0.2]
\tikzstyle{every node}+=[inner sep=0pt]
\node[draw,align=left,inner sep=3pt] at (0,0) {$01$};
\draw [black] (56.6,-16.1) circle (3);
\draw [black] (72.2,-16.1) circle (3);
\draw [black] (72.2,-16.1) circle (2.4);
\draw [black] (43.4,-16.1) circle (3);
\draw [black] (28.2,-16.1) circle (3);
\draw [black] (46.4,-16.1) -- (53.6,-16.1);
\fill [black] (53.6,-16.1) -- (52.8,-15.6) -- (52.8,-16.6);
\draw (50,-16.6) node [below] {$\epsilon$};
\draw [black] (59.6,-16.1) -- (69.2,-16.1);
\fill [black] (69.2,-16.1) -- (68.4,-15.6) -- (68.4,-16.6);
\draw (64.4,-16.6) node [below] {$1$};
\draw [black] (31.2,-16.1) -- (40.4,-16.1);
\fill [black] (40.4,-16.1) -- (39.6,-15.6) -- (39.6,-16.6);
\draw (35.8,-16.6) node [below] {$0$};
\draw [black] (19.7,-16.1) -- (25.2,-16.1);
\fill [black] (25.2,-16.1) -- (24.4,-15.6) -- (24.4,-16.6);
\end{tikzpicture}


\begin{tikzpicture}[scale=0.2]
\tikzstyle{every node}+=[inner sep=0pt]
\node[draw,align=left,inner sep=3pt] at (0,0) {$(0\cup1)^*$};
\draw [black] (30.6,-7.6) circle (3);
\draw [black] (30.6,-7.6) circle (2.4);
\draw [black] (52.5,-12.2) circle (3);
\draw [black] (42.3,-7.6) circle (3);
\draw [black] (52.5,-3.1) circle (3);
\draw [black] (64.1,-3.1) circle (3);
\draw [black] (64.1,-3.1) circle (2.4);
\draw [black] (64.1,-12.2) circle (3);
\draw [black] (64.1,-12.2) circle (2.4);
\draw [black] (33.6,-7.6) -- (39.3,-7.6);
\fill [black] (39.3,-7.6) -- (38.5,-7.1) -- (38.5,-8.1);
\draw (36.45,-8.1) node [below] {$\epsilon$};
\draw [black] (45.04,-6.39) -- (49.76,-4.31);
\fill [black] (49.76,-4.31) -- (48.82,-4.18) -- (49.23,-5.09);
\draw (48.3,-5.86) node [below] {$\epsilon$};
\draw [black] (45.03,-8.83) -- (49.77,-10.97);
\fill [black] (49.77,-10.97) -- (49.24,-10.18) -- (48.83,-11.09);
\draw (46.5,-10.41) node [below] {$\epsilon$};
\draw [black] (55.5,-3.1) -- (61.1,-3.1);
\fill [black] (61.1,-3.1) -- (60.3,-2.6) -- (60.3,-3.6);
\draw (58.3,-3.6) node [below] {$0$};
\draw [black] (55.5,-12.2) -- (61.1,-12.2);
\fill [black] (61.1,-12.2) -- (60.3,-11.7) -- (60.3,-12.7);
\draw (58.3,-12.7) node [below] {$1$};
\draw [black] (43.332,-4.791) arc (152.826:50.50059:12.262);
\fill [black] (43.33,-4.79) -- (44.14,-4.31) -- (43.25,-3.85);
\draw (51.22,2.2) node [above] {$\epsilon$};
\draw [black] (61.974,-14.307) arc (-52.16312:-151.66718:12.44);
\fill [black] (43.39,-10.39) -- (43.33,-11.33) -- (44.21,-10.85);
\draw (51.22,-17.23) node [below] {$\epsilon$};
\draw [black] (22.7,-7.6) -- (27.6,-7.6);
\fill [black] (27.6,-7.6) -- (26.8,-7.1) -- (26.8,-8.1);
\end{tikzpicture}


\begin{tikzpicture}[scale=0.2]
\tikzstyle{every node}+=[inner sep=0pt]
\node[draw,align=left,inner sep=3pt] at (0,0) {$0(0\cup1)^*$};
\draw [black] (8.2,-7.6) circle (3);
\draw [black] (18.6,-7.6) circle (3);
\draw [black] (30.6,-7.6) circle (3);
\draw [black] (30.6,-7.6) circle (2.4);
\draw [black] (52.5,-12.2) circle (3);
\draw [black] (42.3,-7.6) circle (3);
\draw [black] (52.5,-3.1) circle (3);
\draw [black] (64.1,-3.1) circle (3);
\draw [black] (64.1,-3.1) circle (2.4);
\draw [black] (64.1,-12.2) circle (3);
\draw [black] (64.1,-12.2) circle (2.4);
\draw [black] (2.1,-7.6) -- (5.2,-7.6);
\fill [black] (5.2,-7.6) -- (4.4,-7.1) -- (4.4,-8.1);
\draw [black] (11.2,-7.6) -- (15.6,-7.6);
\fill [black] (15.6,-7.6) -- (14.8,-7.1) -- (14.8,-8.1);
\draw (13.4,-8.1) node [below] {$0$};
\draw [black] (21.6,-7.6) -- (27.6,-7.6);
\fill [black] (27.6,-7.6) -- (26.8,-7.1) -- (26.8,-8.1);
\draw (24.6,-8.1) node [below] {$\epsilon$};
\draw [black] (33.6,-7.6) -- (39.3,-7.6);
\fill [black] (39.3,-7.6) -- (38.5,-7.1) -- (38.5,-8.1);
\draw (36.45,-8.1) node [below] {$\epsilon$};
\draw [black] (45.04,-6.39) -- (49.76,-4.31);
\fill [black] (49.76,-4.31) -- (48.82,-4.18) -- (49.23,-5.09);
\draw (48.3,-5.86) node [below] {$\epsilon$};
\draw [black] (45.03,-8.83) -- (49.77,-10.97);
\fill [black] (49.77,-10.97) -- (49.24,-10.18) -- (48.83,-11.09);
\draw (46.5,-10.41) node [below] {$\epsilon$};
\draw [black] (55.5,-3.1) -- (61.1,-3.1);
\fill [black] (61.1,-3.1) -- (60.3,-2.6) -- (60.3,-3.6);
\draw (58.3,-3.6) node [below] {$0$};
\draw [black] (55.5,-12.2) -- (61.1,-12.2);
\fill [black] (61.1,-12.2) -- (60.3,-11.7) -- (60.3,-12.7);
\draw (58.3,-12.7) node [below] {$1$};
\draw [black] (43.332,-4.791) arc (152.826:50.50059:12.262);
\fill [black] (43.33,-4.79) -- (44.14,-4.31) -- (43.25,-3.85);
\draw (51.22,2.2) node [above] {$\epsilon$};
\draw [black] (61.974,-14.307) arc (-52.16312:-151.66718:12.44);
\fill [black] (43.39,-10.39) -- (43.33,-11.33) -- (44.21,-10.85);
\draw (51.22,-17.23) node [below] {$\epsilon$};
\end{tikzpicture}


\begin{tikzpicture}[scale=0.2]
\tikzstyle{every node}+=[inner sep=0pt]
\node[draw,align=left,inner sep=3pt] at (0,0) {$0(0\cup1)^*01$};
\draw [black] (8.2,-7.6) circle (3);
\draw [black] (18.6,-7.6) circle (3);
\draw [black] (30.6,-7.6) circle (3);
\draw [black] (52.5,-12.2) circle (3);
\draw [black] (42.3,-7.6) circle (3);
\draw [black] (52.5,-3.1) circle (3);
\draw [black] (64.1,-3.1) circle (3);
\draw [black] (64.1,-12.2) circle (3);
\draw [black] (64.1,-24.8) circle (3);
\draw [black] (39.3,-27.3) circle (3);
\draw [black] (23.5,-27.3) circle (3);
\draw [black] (8.2,-27.3) circle (3);
\draw [black] (8.2,-27.3) circle (2.4);
\draw [black] (2.1,-7.6) -- (5.2,-7.6);
\fill [black] (5.2,-7.6) -- (4.4,-7.1) -- (4.4,-8.1);
\draw [black] (11.2,-7.6) -- (15.6,-7.6);
\fill [black] (15.6,-7.6) -- (14.8,-7.1) -- (14.8,-8.1);
\draw (13.4,-8.1) node [below] {$0$};
\draw [black] (21.6,-7.6) -- (27.6,-7.6);
\fill [black] (27.6,-7.6) -- (26.8,-7.1) -- (26.8,-8.1);
\draw (24.6,-8.1) node [below] {$\epsilon$};
\draw [black] (33.6,-7.6) -- (39.3,-7.6);
\fill [black] (39.3,-7.6) -- (38.5,-7.1) -- (38.5,-8.1);
\draw (36.45,-8.1) node [below] {$\epsilon$};
\draw [black] (45.04,-6.39) -- (49.76,-4.31);
\fill [black] (49.76,-4.31) -- (48.82,-4.18) -- (49.23,-5.09);
\draw (48.3,-5.86) node [below] {$\epsilon$};
\draw [black] (45.03,-8.83) -- (49.77,-10.97);
\fill [black] (49.77,-10.97) -- (49.24,-10.18) -- (48.83,-11.09);
\draw (46.5,-10.41) node [below] {$\epsilon$};
\draw [black] (55.5,-3.1) -- (61.1,-3.1);
\fill [black] (61.1,-3.1) -- (60.3,-2.6) -- (60.3,-3.6);
\draw (58.3,-3.6) node [below] {$0$};
\draw [black] (55.5,-12.2) -- (61.1,-12.2);
\fill [black] (61.1,-12.2) -- (60.3,-11.7) -- (60.3,-12.7);
\draw (58.3,-12.7) node [below] {$1$};
\draw [black] (43.332,-4.791) arc (152.826:50.50059:12.262);
\fill [black] (43.33,-4.79) -- (44.14,-4.31) -- (43.25,-3.85);
\draw (51.22,2.2) node [above] {$\epsilon$};
\draw [black] (61.974,-14.307) arc (-52.16312:-151.66718:12.44);
\fill [black] (43.39,-10.39) -- (43.33,-11.33) -- (44.21,-10.85);
\draw (51.22,-17.23) node [below] {$\epsilon$};
\draw [black] (66.952,-4.002) arc (64.64247:-64.64247:11.009);
\fill [black] (66.95,-23.9) -- (67.89,-24.01) -- (67.46,-23.1);
\draw (73.75,-13.95) node [right] {$\epsilon$};
\draw [black] (64.1,-15.2) -- (64.1,-21.8);
\fill [black] (64.1,-21.8) -- (64.6,-21) -- (63.6,-21);
\draw (63.6,-18.5) node [left] {$\epsilon$};
\draw [black] (61.12,-25.1) -- (42.28,-27);
\fill [black] (42.28,-27) -- (43.13,-27.42) -- (43.03,-26.42);
\draw (51.51,-25.47) node [above] {$0$};
\draw [black] (36.3,-27.3) -- (26.5,-27.3);
\fill [black] (26.5,-27.3) -- (27.3,-27.8) -- (27.3,-26.8);
\draw (31.4,-26.8) node [above] {$\epsilon$};
\draw [black] (20.5,-27.3) -- (11.2,-27.3);
\fill [black] (11.2,-27.3) -- (12,-27.8) -- (12,-26.8);
\draw (15.85,-26.8) node [above] {$1$};
\draw [black] (61.136,-24.338) arc (-100.40323:-133.95168:55.375);
\fill [black] (61.14,-24.34) -- (60.44,-23.7) -- (60.26,-24.69);
\draw (44.93,-19.64) node [below] {$\epsilon$};
\end{tikzpicture}


\begin{tikzpicture}[scale=0.2]
\tikzstyle{every node}+=[inner sep=0pt]
\node[draw,align=left,inner sep=3pt] at (0,0) {$0(0\cup1)^*01(0\cup1)^*$};
\draw [black] (8.2,-7.6) circle (3);
\draw [black] (18.6,-7.6) circle (3);
\draw [black] (30.6,-7.6) circle (3);
\draw [black] (52.5,-12.2) circle (3);
\draw [black] (42.3,-7.6) circle (3);
\draw [black] (52.5,-3.1) circle (3);
\draw [black] (64.1,-3.1) circle (3);
\draw [black] (64.1,-12.2) circle (3);
\draw [black] (70.9,-28.6) circle (3);
\draw [black] (53.3,-31.9) circle (3);
\draw [black] (33.3,-25.6) circle (3);
\draw [black] (20.6,-25.6) circle (3);
\draw [black] (8.2,-25.6) circle (3);
\draw [black] (8.2,-25.6) circle (2.4);
\draw [black] (8.2,-47) circle (3);
\draw [black] (20.6,-41.7) circle (3);
\draw [black] (20.6,-53.6) circle (3);
\draw [black] (35,-41.7) circle (3);
\draw [black] (35,-41.7) circle (2.4);
\draw [black] (35,-53.6) circle (3);
\draw [black] (35,-53.6) circle (2.4);
\draw [black] (2.1,-7.6) -- (5.2,-7.6);
\fill [black] (5.2,-7.6) -- (4.4,-7.1) -- (4.4,-8.1);
\draw [black] (11.2,-7.6) -- (15.6,-7.6);
\fill [black] (15.6,-7.6) -- (14.8,-7.1) -- (14.8,-8.1);
\draw (13.4,-8.1) node [below] {$0$};
\draw [black] (21.6,-7.6) -- (27.6,-7.6);
\fill [black] (27.6,-7.6) -- (26.8,-7.1) -- (26.8,-8.1);
\draw (24.6,-8.1) node [below] {$\epsilon$};
\draw [black] (33.6,-7.6) -- (39.3,-7.6);
\fill [black] (39.3,-7.6) -- (38.5,-7.1) -- (38.5,-8.1);
\draw (36.45,-8.1) node [below] {$\epsilon$};
\draw [black] (45.04,-6.39) -- (49.76,-4.31);
\fill [black] (49.76,-4.31) -- (48.82,-4.18) -- (49.23,-5.09);
\draw (48.3,-5.86) node [below] {$\epsilon$};
\draw [black] (45.03,-8.83) -- (49.77,-10.97);
\fill [black] (49.77,-10.97) -- (49.24,-10.18) -- (48.83,-11.09);
\draw (46.5,-10.41) node [below] {$\epsilon$};
\draw [black] (55.5,-3.1) -- (61.1,-3.1);
\fill [black] (61.1,-3.1) -- (60.3,-2.6) -- (60.3,-3.6);
\draw (58.3,-3.6) node [below] {$0$};
\draw [black] (55.5,-12.2) -- (61.1,-12.2);
\fill [black] (61.1,-12.2) -- (60.3,-11.7) -- (60.3,-12.7);
\draw (58.3,-12.7) node [below] {$1$};
\draw [black] (43.332,-4.791) arc (152.826:50.50059:12.262);
\fill [black] (43.33,-4.79) -- (44.14,-4.31) -- (43.25,-3.85);
\draw (51.22,2.2) node [above] {$\epsilon$};
\draw [black] (61.974,-14.307) arc (-52.16312:-151.66718:12.44);
\fill [black] (43.39,-10.39) -- (43.33,-11.33) -- (44.21,-10.85);
\draw (51.22,-17.23) node [below] {$\epsilon$};
\draw [black] (67.032,-3.707) arc (72.21038:-42.34754:14.09);
\fill [black] (73.14,-26.61) -- (74.05,-26.36) -- (73.31,-25.69);
\draw (77.11,-12.99) node [right] {$\epsilon$};
\draw [black] (65.25,-14.97) -- (69.75,-25.83);
\fill [black] (69.75,-25.83) -- (69.91,-24.9) -- (68.98,-25.28);
\draw (66.76,-21.32) node [left] {$\epsilon$};
\draw [black] (67.95,-29.15) -- (56.25,-31.35);
\fill [black] (56.25,-31.35) -- (57.13,-31.69) -- (56.94,-30.71);
\draw (61.58,-29.66) node [above] {$0$};
\draw [black] (50.44,-31) -- (36.16,-26.5);
\fill [black] (36.16,-26.5) -- (36.77,-27.22) -- (37.07,-26.26);
\draw (44.09,-28.21) node [above] {$\epsilon$};
\draw [black] (30.3,-25.6) -- (23.6,-25.6);
\fill [black] (23.6,-25.6) -- (24.4,-26.1) -- (24.4,-25.1);
\draw (26.95,-25.1) node [above] {$1$};
\draw [black] (17.6,-25.6) -- (11.2,-25.6);
\fill [black] (11.2,-25.6) -- (12,-26.1) -- (12,-25.1);
\draw (14.4,-25.1) node [above] {$\epsilon$};
\draw [black] (10.96,-45.82) -- (17.84,-42.88);
\fill [black] (17.84,-42.88) -- (16.91,-42.73) -- (17.3,-43.65);
\draw (15.29,-44.86) node [below] {$\epsilon$};
\draw [black] (10.85,-48.41) -- (17.95,-52.19);
\fill [black] (17.95,-52.19) -- (17.48,-51.37) -- (17.01,-52.26);
\draw (13.48,-50.8) node [below] {$\epsilon$};
\draw [black] (23.6,-41.7) -- (32,-41.7);
\fill [black] (32,-41.7) -- (31.2,-41.2) -- (31.2,-42.2);
\draw (27.8,-42.2) node [below] {$0$};
\draw [black] (23.6,-53.6) -- (32,-53.6);
\fill [black] (32,-53.6) -- (31.2,-53.1) -- (31.2,-54.1);
\draw (27.8,-54.1) node [below] {$1$};
\draw [black] (9.523,-44.312) arc (148.46843:53.90464:16.116);
\fill [black] (9.52,-44.31) -- (10.37,-43.89) -- (9.51,-43.37);
\draw (19.62,-36.35) node [above] {$\epsilon$};
\draw [black] (32.928,-55.763) arc (-49.43995:-158.22976:15.136);
\fill [black] (9.03,-49.88) -- (8.86,-50.81) -- (9.79,-50.44);
\draw (18.81,-59.53) node [below] {$\epsilon$};
\draw [black] (5.978,-44.994) arc (-138.61611:-221.38389:13.151);
\fill [black] (5.98,-44.99) -- (5.82,-44.06) -- (5.07,-44.72);
\draw (2.19,-36.3) node [left] {$\epsilon$};
\draw [black] (67.966,-27.973) arc (-103.14634:-131.90096:79.862);
\fill [black] (67.97,-27.97) -- (67.3,-27.3) -- (67.07,-28.28);
\draw (48.31,-21.53) node [below] {$\epsilon$};
\end{tikzpicture}


\begin{tikzpicture}[scale=0.2]
\tikzstyle{every node}+=[inner sep=0pt]
\node[draw,align=left,inner sep=3pt] at (0,0) {$0(0\cup1)^*01(0\cup1)^*1$};
\draw [black] (8.2,-7.6) circle (3);
\draw [black] (18.6,-7.6) circle (3);
\draw [black] (30.6,-7.6) circle (3);
\draw [black] (52.5,-12.2) circle (3);
\draw [black] (42.3,-7.6) circle (3);
\draw [black] (52.5,-3.1) circle (3);
\draw [black] (64.1,-3.1) circle (3);
\draw [black] (64.1,-12.2) circle (3);
\draw [black] (70.9,-28.6) circle (3);
\draw [black] (53.3,-31.9) circle (3);
\draw [black] (33.3,-25.6) circle (3);
\draw [black] (20.6,-25.6) circle (3);
\draw [black] (8.2,-25.6) circle (3);
\draw [black] (8.2,-47) circle (3);
\draw [black] (20.6,-41.7) circle (3);
\draw [black] (20.6,-53.6) circle (3);
\draw [black] (35,-41.7) circle (3);
\draw [black] (35,-53.6) circle (3);
\draw [black] (48.5,-47) circle (3);
\draw [black] (63.5,-47) circle (3);
\draw [black] (63.5,-47) circle (2.4);
\draw [black] (2.1,-7.6) -- (5.2,-7.6);
\fill [black] (5.2,-7.6) -- (4.4,-7.1) -- (4.4,-8.1);
\draw [black] (11.2,-7.6) -- (15.6,-7.6);
\fill [black] (15.6,-7.6) -- (14.8,-7.1) -- (14.8,-8.1);
\draw (13.4,-8.1) node [below] {$0$};
\draw [black] (21.6,-7.6) -- (27.6,-7.6);
\fill [black] (27.6,-7.6) -- (26.8,-7.1) -- (26.8,-8.1);
\draw (24.6,-8.1) node [below] {$\epsilon$};
\draw [black] (33.6,-7.6) -- (39.3,-7.6);
\fill [black] (39.3,-7.6) -- (38.5,-7.1) -- (38.5,-8.1);
\draw (36.45,-8.1) node [below] {$\epsilon$};
\draw [black] (45.04,-6.39) -- (49.76,-4.31);
\fill [black] (49.76,-4.31) -- (48.82,-4.18) -- (49.23,-5.09);
\draw (48.3,-5.86) node [below] {$\epsilon$};
\draw [black] (45.03,-8.83) -- (49.77,-10.97);
\fill [black] (49.77,-10.97) -- (49.24,-10.18) -- (48.83,-11.09);
\draw (46.5,-10.41) node [below] {$\epsilon$};
\draw [black] (55.5,-3.1) -- (61.1,-3.1);
\fill [black] (61.1,-3.1) -- (60.3,-2.6) -- (60.3,-3.6);
\draw (58.3,-3.6) node [below] {$0$};
\draw [black] (55.5,-12.2) -- (61.1,-12.2);
\fill [black] (61.1,-12.2) -- (60.3,-11.7) -- (60.3,-12.7);
\draw (58.3,-12.7) node [below] {$1$};
\draw [black] (43.332,-4.791) arc (152.826:50.50059:12.262);
\fill [black] (43.33,-4.79) -- (44.14,-4.31) -- (43.25,-3.85);
\draw (51.22,2.2) node [above] {$\epsilon$};
\draw [black] (61.974,-14.307) arc (-52.16312:-151.66718:12.44);
\fill [black] (43.39,-10.39) -- (43.33,-11.33) -- (44.21,-10.85);
\draw (51.22,-17.23) node [below] {$\epsilon$};
\draw [black] (67.032,-3.707) arc (72.21038:-42.34754:14.09);
\fill [black] (73.14,-26.61) -- (74.05,-26.36) -- (73.31,-25.69);
\draw (77.11,-12.99) node [right] {$\epsilon$};
\draw [black] (65.25,-14.97) -- (69.75,-25.83);
\fill [black] (69.75,-25.83) -- (69.91,-24.9) -- (68.98,-25.28);
\draw (66.76,-21.32) node [left] {$\epsilon$};
\draw [black] (67.95,-29.15) -- (56.25,-31.35);
\fill [black] (56.25,-31.35) -- (57.13,-31.69) -- (56.94,-30.71);
\draw (61.58,-29.66) node [above] {$0$};
\draw [black] (50.44,-31) -- (36.16,-26.5);
\fill [black] (36.16,-26.5) -- (36.77,-27.22) -- (37.07,-26.26);
\draw (44.09,-28.21) node [above] {$\epsilon$};
\draw [black] (30.3,-25.6) -- (23.6,-25.6);
\fill [black] (23.6,-25.6) -- (24.4,-26.1) -- (24.4,-25.1);
\draw (26.95,-25.1) node [above] {$1$};
\draw [black] (17.6,-25.6) -- (11.2,-25.6);
\fill [black] (11.2,-25.6) -- (12,-26.1) -- (12,-25.1);
\draw (14.4,-25.1) node [above] {$\epsilon$};
\draw [black] (10.96,-45.82) -- (17.84,-42.88);
\fill [black] (17.84,-42.88) -- (16.91,-42.73) -- (17.3,-43.65);
\draw (15.29,-44.86) node [below] {$\epsilon$};
\draw [black] (10.85,-48.41) -- (17.95,-52.19);
\fill [black] (17.95,-52.19) -- (17.48,-51.37) -- (17.01,-52.26);
\draw (13.48,-50.8) node [below] {$\epsilon$};
\draw [black] (23.6,-41.7) -- (32,-41.7);
\fill [black] (32,-41.7) -- (31.2,-41.2) -- (31.2,-42.2);
\draw (27.8,-42.2) node [below] {$0$};
\draw [black] (23.6,-53.6) -- (32,-53.6);
\fill [black] (32,-53.6) -- (31.2,-53.1) -- (31.2,-54.1);
\draw (27.8,-54.1) node [below] {$1$};
\draw [black] (9.523,-44.312) arc (148.46843:53.90464:16.116);
\fill [black] (9.52,-44.31) -- (10.37,-43.89) -- (9.51,-43.37);
\draw (19.62,-36.35) node [above] {$\epsilon$};
\draw [black] (32.928,-55.763) arc (-49.43995:-158.22976:15.136);
\fill [black] (9.03,-49.88) -- (8.86,-50.81) -- (9.79,-50.44);
\draw (18.81,-59.53) node [below] {$\epsilon$};
\draw [black] (37.79,-42.8) -- (45.71,-45.9);
\fill [black] (45.71,-45.9) -- (45.15,-45.15) -- (44.78,-46.08);
\draw (40.88,-44.87) node [below] {$\epsilon$};
\draw [black] (37.7,-52.28) -- (45.8,-48.32);
\fill [black] (45.8,-48.32) -- (44.87,-48.22) -- (45.31,-49.12);
\draw (42.66,-50.8) node [below] {$\epsilon$};
\draw [black] (51.5,-47) -- (60.5,-47);
\fill [black] (60.5,-47) -- (59.7,-46.5) -- (59.7,-47.5);
\draw (56,-47.5) node [below] {$1$};
\draw [black] (5.978,-44.994) arc (-138.61611:-221.38389:13.151);
\fill [black] (5.98,-44.99) -- (5.82,-44.06) -- (5.07,-44.72);
\draw (2.19,-36.3) node [left] {$\epsilon$};
\draw [black] (67.966,-27.973) arc (-103.14634:-131.90096:79.862);
\fill [black] (67.97,-27.97) -- (67.3,-27.3) -- (67.07,-28.28);
\draw (48.31,-21.53) node [below] {$\epsilon$};
\draw [black] (11.077,-26.45) arc (72.74568:51.31616:106.902);
\fill [black] (46.18,-45.09) -- (45.87,-44.2) -- (45.25,-44.98);
\draw (30.42,-33.62) node [above] {$\epsilon$};
\end{tikzpicture}


\noindent \textbf{L3.4} No autômato generalizado da figura 1.67b, foi removido o estado 2, resultando no autômato da figura 1.67c. Foi então removido o estado 1 para produzir o autômato da figura 1.67d, obtendo-se assim uma expressão regular final. Refazer as contas produzindo um autômato generalizado ao se remover o estado 1 daquele da figura 1.67b. Deste autômato generalizado, remova o estado 2 e produza um novo autômato generalizado final com dois estados.

\textbf{Resposta:} 

\begin{center}
\begin{tikzpicture}[scale=0.2]
\tikzstyle{every node}+=[inner sep=0pt]
\draw [black] (7.9,-19.1) circle (3);
\draw (7.9,-19.1) node {$1$};
\draw [black] (7.9,-37.1) circle (3);
\draw (7.9,-37.1) node {$2$};
\draw [black] (7.9,-37.1) circle (2.4);
\draw [black] (0.8,-19.1) -- (4.9,-19.1);
\fill [black] (4.9,-19.1) -- (4.1,-18.6) -- (4.1,-19.6);
\draw [black] (7.9,-22.1) -- (7.9,-34.1);
\fill [black] (7.9,-34.1) -- (8.4,-33.3) -- (7.4,-33.3);
\draw (7.4,-28.1) node [left] {$b$};
\draw [black] (10.58,-17.777) arc (144:-144:2.25);
\draw (15.15,-19.1) node [right] {$a$};
\fill [black] (10.58,-20.42) -- (10.93,-21.3) -- (11.52,-20.49);
\draw [black] (10.399,-35.462) arc (150.971:-137.029:2.25);
\draw (15.27,-36.15) node [right] {$a,\mbox{ }b$};
\fill [black] (10.72,-38.09) -- (11.18,-38.91) -- (11.66,-38.04);
\end{tikzpicture}
\end{center}

\begin{center}
\begin{tikzpicture}[scale=0.2]
\tikzstyle{every node}+=[inner sep=0pt]
\draw [black] (39.5,-20.5) circle (3);
\draw (39.5,-20.5) node {$1$};
\draw [black] (39.5,-36.1) circle (3);
\draw (39.5,-36.1) node {$2$};
\draw [black] (23.3,-20.5) circle (3);
\draw (23.3,-20.5) node {$s$};
\draw [black] (22.7,-36.1) circle (3);
\draw (22.7,-36.1) node {$\alpha$};
\draw [black] (22.7,-36.1) circle (2.4);
\draw [black] (39.5,-23.5) -- (39.5,-33.1);
\fill [black] (39.5,-33.1) -- (40,-32.3) -- (39,-32.3);
\draw (39,-28.3) node [left] {$b$};
\draw [black] (42.18,-19.177) arc (144:-144:2.25);
\draw (46.75,-20.5) node [right] {$a$};
\fill [black] (42.18,-21.82) -- (42.53,-22.7) -- (43.12,-21.89);
\draw [black] (41.999,-34.462) arc (150.971:-137.029:2.25);
\draw (46.87,-35.15) node [right] {$a\mbox{ }\cup\mbox{ }b$};
\fill [black] (42.32,-37.09) -- (42.78,-37.91) -- (43.26,-37.04);
\draw [black] (16.5,-20.5) -- (20.3,-20.5);
\fill [black] (20.3,-20.5) -- (19.5,-20) -- (19.5,-21);
\draw [black] (26.3,-20.5) -- (36.5,-20.5);
\fill [black] (36.5,-20.5) -- (35.7,-20) -- (35.7,-21);
\draw (31.4,-21) node [below] {$\epsilon$};
\draw [black] (36.5,-36.1) -- (25.7,-36.1);
\fill [black] (25.7,-36.1) -- (26.5,-36.6) -- (26.5,-35.6);
\draw (31.1,-35.6) node [above] {$\epsilon$};
\end{tikzpicture}
\end{center}

Removendo o estado 1, onde $q_{rem} = 1, q_i = s, q_j = 2, R_1 = \epsilon, R_2 = a, R_3 = b, R4 = \emptyset$, temos $(\epsilon)(a)^*(b)\cup(\emptyset) = a^*b$:
\begin{center}
\begin{tikzpicture}[scale=0.2]
\tikzstyle{every node}+=[inner sep=0pt]
\draw [black] (39.5,-36.1) circle (3);
\draw (39.5,-36.1) node {$2$};
\draw [black] (23.3,-20.5) circle (3);
\draw (23.3,-20.5) node {$s$};
\draw [black] (22.7,-36.1) circle (3);
\draw (22.7,-36.1) node {$\alpha$};
\draw [black] (22.7,-36.1) circle (2.4);
\draw [black] (41.999,-34.462) arc (150.971:-137.029:2.25);
\draw (46.87,-35.15) node [right] {$a \cup b$};
\fill [black] (42.32,-37.09) -- (42.78,-37.91) -- (43.26,-37.04);
\draw [black] (16.5,-20.5) -- (20.3,-20.5);
\fill [black] (20.3,-20.5) -- (19.5,-20) -- (19.5,-21);
\draw [black] (36.5,-36.1) -- (25.7,-36.1);
\fill [black] (25.7,-36.1) -- (26.5,-36.6) -- (26.5,-35.6);
\draw (31.1,-35.6) node [above] {$\epsilon$};
\draw [black] (25.46,-22.58) -- (37.34,-34.02);
\fill [black] (37.34,-34.02) -- (37.11,-33.1) -- (36.42,-33.82);
\draw (28.97,-28.78) node [below] {$a^*b$};
\end{tikzpicture}
\end{center}

Removendo o estado 2, onde $q_{rem} = 2, q_i = s, q_j = \alpha, R_1 = a^*b, R_2 = a \cup b, R_3 = \epsilon, R4 = \emptyset$, temos $(a^*b) (a \cup b)^* (\epsilon)\cup(\emptyset) = (a^*b) (a \cup b)^*$:
\begin{center}
\begin{tikzpicture}[scale=0.2]
\tikzstyle{every node}+=[inner sep=0pt]
\draw [black] (23.3,-20.5) circle (3);
\draw (23.3,-20.5) node {$s$};
\draw [black] (23.3,-36.1) circle (3);
\draw (23.3,-36.1) node {$\alpha$};
\draw [black] (23.3,-36.1) circle (2.4);
\draw [black] (16.5,-20.5) -- (20.3,-20.5);
\fill [black] (20.3,-20.5) -- (19.5,-20) -- (19.5,-21);
\draw [black] (23.3,-23.5) -- (23.3,-33.1);
\fill [black] (23.3,-33.1) -- (23.8,-32.3) -- (22.8,-32.3);
\draw (23.8,-28.3) node [right] {$(a^*b)(a \cup b)^*$};
\end{tikzpicture}
\end{center}

\begin{center}
\textbf{\large{Lista 4}}
\end{center}

\noindent \textbf{L4.1} Complete a demonstração formal do lema 2.21, a primeira parte do teorema 2.20. A saber, primeiro demonstre que, para toda palavra $w$ derivada pela gramática $A$, uma computação que aceite a palavra $w$ no autômato construído $P$ pode conduzir do estado $q_{inicio}$ para o estado $q_{aceita}$. Em seguida, demonstre que toda palavra $w$ aceita por uma computação de $P$ admite uma derivação pela gramática $A$. \\[3pt]
\textbf{Resposta: } Seja $L$ a linguagem livre de contexto em questão e seja $A = (V, \Sigma, R, S)$ uma gramática que gera $L$.

Seja $P$ o autômato construído na prova do lema 2.21. O autômato construído possui, basicamente, três "macroestados": o estado de início, o estado de loop e um estado final.

Para demostrar que o autômato a pilha construído no lema 2.21 reconhece a mesma linguagem que é associada à gramática a partir da qual o autômato a pilha foi construído, vamos enunciar a seguinte afirmação:

\textsc{Afirmação:} Seja $w$ uma palavra composta apenas por terminais, ou seja, $w \in \Sigma^*$. Seja $\alpha = \epsilon$ ou uma concatenação qualquer de variáveis e terminais de $A$ começando com uma variável. Então, partindo da variável de início $S$ de $A$, conseguimos gerar $w\alpha$ por uma sucessão de regras de $A$ se, e somente se, existe uma computação do autômato $P$ que vai do estado inicial e passa pelo estado de loop após ter processado um trecho $w$ da palavra de entrada e, precisamente, com $\alpha\$$ na pilha.

A afirmação acima é mais forte do que queremos provar, que seria apenas o caso de $\alpha$ ser a palavra vazia, pois implicaria para qualquer palavra $w \in \Sigma^*$, partindo da variável inicial $S$ de $A$, conseguimos gerar $w$ por uma sucessão de regras de $A$ se, e somente se, existe uma computação do autômato $P$ que vai do estado inicial e passa pelo estado de loop após ter processado $w$ e, precisamente, com $\$$ na pilha. Neste caso, a transição $(\epsilon, \$ \rightarrow \epsilon)$ nos levaria ao estado final do autômato, aceitando, assim, a palavra $w$.

Dessa forma, vamos apenas demonstrar a afirmação e, para tanto, vamos dividir a prova em duas partes, a saber:

Seja $w$ uma palavra composta apenas por terminais, ou seja, $w \in \Sigma^*$. Seja $\alpha = \epsilon$ ou uma concatenação qualquer de variáveis e terminais de $A$ começando com uma variável.

\textbf{1.} Partindo da variável inicial $S$ de $A$, conseguimos gerar $w\alpha$ por uma sucessão de regras de $A$, se existe uma computação do autômato $P$ que vai do estado inicial e passa pelo estado de loop após ter processado um trecho w da palavra de entrada e, precisamente, com $\alpha\$$ na pilha.
\begin{proof} Prova por indução no comprimento $k$ de uma derivação mais à esquerda de $w$ a partir de $S$.

\indbase Para $k = 0$ não temos qualquer derivação. Logo, $w = \epsilon$ e $\alpha = S$ é o início da derivação e $S\$$ está na pilha no início do processamento.

\indhypo Suponha que a a afirmação em \textbf{1.} é válida $\forall k \geq 0$.

\indstep Seja $w$ uma cadeia, onde $w \in \Sigma^*$ seguida de $\alpha$ que é uma concatenação qualquer de variáveis e terminais de $A$ começando com uma variável, tal que $w\alpha$ foi obtida após $k + 1$ derivações mais à esquerda a partir de $S$. Seja $S \Rightarrow u_0 \Rightarrow u_1 \ldots \Rightarrow u_{k-1} \Rightarrow u_k$ essas $k + 1$ derivações a partir de $S$ que geram $w\alpha$. A $k$-ésima derivação pode ser descrita como $Y \rightarrow \delta$, e $u_k$ (resultado da $k$-ésima derivação) deve ser da forma $xY\beta$, tal que $x \in \Sigma^*$, $Y \in V$ e $\beta = V(V \cup \Sigma)^*$, já que as derivações são sempre mais à esquerda, a derivação a ser usada será $Y \rightarrow \delta$ e, ainda há $\alpha$ a ser processado.

Pela hipótese de indução, temos que existe uma computação no autômato $P$ que vai do estado inicial e passa pelo estado de loop após ter processado um trecho $w' = x$ (trecho de $w$) da palavra de entrada e, precisamente, com $\alpha'\$$ na pilha, tal que $\alpha' = Y\beta$. Fazendo, pois, a $(k+1)$-ésima derivação, temos na pilha uma troca de $Y$ por $\delta$, ou seja, $xY\beta \Rightarrow x\delta\beta$. Como $x\delta\beta = w\alpha \neq w'\alpha'$, temos que $\delta$ também é da forma $zW\sigma$, onde $z \in \Sigma^*$, $W \in V$ e $\sigma = V(V \cup \Sigma)^*$, ou seja, $x\delta\beta \Rightarrow xzW\sigma\beta = w\alpha$. Os símbolos terminais são sempre consumidos e retirados da pilha pelo laço do autômato $P$, restando na pilha $W\sigma\beta$. Dessa forma, temos que $w = xy$ e $\alpha = W\sigma\beta$ que, juntamente com $\$$ é, precisamente, o que temos na pilha após o processamento de $w$.
\end{proof}

\textbf{2.} Se existe uma computação do autômato $P$ que vai do estado inicial e passa pelo estado de loop após ter processado um trecho $w$ da palavra de entrada e, precisamente, com $\alpha\$$ na pilha, então conseguimos gerar $w\alpha$ por uma sucessão de regras de $A$.
\begin{proof} Prova por indução no número $k$ de vezes que a computação em questão passa por "supertransições" (uma supertransição é qualquer uma daquelas que saem do estado de loop e entram no estado de loop).

\indbase Para $k = 0$, utilizamos apenas a transição $(\epsilon, \epsilon \Rightarrow S\$)$ que parte do estado inicial e chega no estado de loop. Logo, nada de $w$ foi processado, ou seja, $w = \epsilon$ e $\alpha = S$, que é o topo da pilha (onde temos somente $S\$$). Como $S$ é o símbolo inicial da gramática $A$, é sempre possível gerar $w\alpha$ a partir de $A$.

\indhypo Suponha que a a afirmação em \textbf{2.} é válida $\forall k \geq 0$.

\indstep Seja $w \in \Sigma^*$ computada em $P$ após passar por $k + 1$ "supertransições" e seja $\alpha$ a concatenação de variáveis e terminais de $A$ começando com uma variável, seguida de $\$$, o conteúdo da pilha após processar $w$. Seja $l_1, l_2, \ldots l_k, l_{k+1}$ as $k+1$ passagens pelas “supertransições” em $P$. Seja $w' \in \Sigma^*$ a cadeia computada em $P$ após passar por $i \leq k$ transições do tipo "supertransições", tal que tenhamos $\alpha'\$$ na pilha, onde $\alpha'$ é a concatenação de variáveis e terminais de $A$ começando com uma variável. Pela hipótese de indução (não consegui terminar a tempo).
\end{proof}

\noindent \textbf{L4.2 (Sipser 2.9)} Dê uma gramatica livre-do-contexto que gere a linguagem
\begin{center}
$A = \{a^ib^jc^k \ |\ i = j$ ou $j = k$ onde $i, j, k \geq 0\}$
\end{center}

\textbf{Resposta: } A GLC que gera a linguagem $A$ é $G = (\{S, S_1, S_2, A, C\}, \{a, b, c\}, R, S)$, onde $S$ é a variável inicial e $R$ é o conjunto de regras:
\begin{align*}
    S &\rightarrow AS_2 \ |\ S_1C \\
    S_1 &\rightarrow aS_1b \ |\ \epsilon \\
    S_2 &\rightarrow bS_2c \ |\ \epsilon \\
    A &\rightarrow aA \ |\ \epsilon \\
    C &\rightarrow cC \ |\ \epsilon \\
\end{align*}

\textbf{Sua gramática é ambígua? Por que ou por que não?}\\[3pt]
Sim, ela é ambígua, pois $G$ gera uma mesma cadeia, digamos $w$, ambiguamente, ou seja, $w$ tem duas árvores sintáticas distintas. A derivação da cadeia $w = abc$, por exemplo, produz duas árvores sintáticas diferentes.

\begin{figure}[H]
\begin{tikzpicture}[sibling distance=72pt]
\Tree [.$S$ [.$S_1C$ [.$aS_1b$ $ab$ ] [.$cC$ $c$ ] ] ]
\end{tikzpicture}
%
\begin{tikzpicture}[sibling distance=72pt]
\Tree [.$S$ [.$AS_2$ [.$aA$ $a$ ] [.$bS_2c$ $bc$ ] ] ]
\end{tikzpicture}
\centering
\end{figure}


\noindent \textbf{L4.3 (Sipser 2.11)} Converta a GLC $G_4$ do exercício 2.1 para um $AP$ equivalente, usando o teorema 2.20.
\begin{align*}
    E &\rightarrow E + T \ |\ T \\
    T &\rightarrow T \times F \ |\ F \\
    F &\rightarrow (E) \ |\ a \\
\end{align*}

\textbf{Resposta: }

\begin{tikzpicture}[>=stealth',shorten >=1pt,auto,node distance=3.3cm]
  \node[state,initial]      (S)                         {$q_{start}$};
  \node[state]              (q1)    [below of=S]        {1};
  \node[state]              (q2)    [right of=q1]       {2};
  \node[state]              (q3)    [right of=q2]       {3};
  \node[state]              (q4)    [below of=q3]       {4};
  \node[state]              (q5)    [right of=q4]       {5};
  \node[state]              (q6)    [below of=q4]       {6};
  \node[state]              (qloop) [left of=q4]        {$q_{loop}$};
  \node[state]              (q7)    [right of=q6]       {7};
  \node[state,accepting]    (qf)    [left of=q6]       {$q_{accept}$};


  \path[->]
  (S)       edge                node                                {$\epsilon, \epsilon \rightarrow \$$} (q1)
  (q1)      edge                node[sloped, anchor=center, above]  {$\epsilon, \epsilon \rightarrow E$} (qloop)

  (qloop)   edge                node[sloped, anchor=center, above]  {$\epsilon, E \rightarrow T$} (q2)
  (q2)      edge                node                                {$\epsilon, \epsilon \rightarrow +$} (q3)
  (q3)      edge                node[sloped, anchor=center, above]  {$\epsilon, \epsilon \rightarrow E$} (qloop)

  (qloop)   edge                node[sloped, anchor=center, above]  {$\epsilon, T \rightarrow F$} (q4)
  (q4)      edge                node                                {$\epsilon, \epsilon \rightarrow \times$} (q5)
  (q5)      edge[bend right=30]  node[sloped, anchor=center, above]  {$\epsilon, \epsilon \rightarrow T$} (qloop)

  (qloop)   edge                node[sloped, anchor=center, above]  {$\epsilon, F \rightarrow )$} (q6)
  (q6)      edge                node                                {$\epsilon, \epsilon \rightarrow E$} (q7)
  (q7)      edge                node[sloped,anchor=center,above]    {$\epsilon, \epsilon \rightarrow ($} (qloop)

  (qloop)   edge [loop left]    node[align=left]                    {$+, + \rightarrow \epsilon$\\
                                                                     $\times, \times \rightarrow \epsilon$\\
                                                                     $a, a \rightarrow \epsilon$\\
                                                                     $), ) \rightarrow \epsilon$\\
                                                                     $(, ( \rightarrow \epsilon$\\
                                                                     $\epsilon, T \rightarrow F$\\
                                                                     $\epsilon, F \rightarrow a$\\} (qloop)      
  (qloop)   edge                node[sloped,anchor=center,above]    {$\epsilon, \$ \rightarrow \epsilon$} (qf);
\end{tikzpicture}

\noindent \textbf{L4.4 (Sipser 2.14)} Converta a seguinte GLC numa GLC equivalente na forma normal de Chomsky,
usando o procedimento dado no Teorema 2.9.
\begin{align*}
    A &\rightarrow BAB \ |\ B \ |\ \epsilon \\
    B &\rightarrow 00 \ |\ \epsilon
\end{align*}
\textbf{Resposta: } Seguem os passos de acordo com o teorema.
\begin{enumerate}
    \item Nova variável inicial
    \begin{align*}
        S_0 &\rightarrow A \\
        A &\rightarrow BAB \ |\ B \ |\ \epsilon \\
        B &\rightarrow 00 \ |\ \epsilon
    \end{align*}

    \item Removendo a regra $A \rightarrow \epsilon$
    \begin{align*}
        S_0 &\rightarrow A \ |\ \epsilon \\
        A &\rightarrow BAB \ |\ B \ |\ BB \\
        B &\rightarrow 00 \ |\ \epsilon
    \end{align*}

    \item Removendo a regra $B \rightarrow \epsilon$
    \begin{align*}
        S_0 &\rightarrow A \ |\ \epsilon \\
        A &\rightarrow BAB \ |\ B \ |\ BB \ |\ AB \ |\ BA \\
        B &\rightarrow 00
    \end{align*}

    \item Removendo a regra unitária $A \rightarrow B$
    \begin{align*}
        S_0 &\rightarrow A \ |\ \epsilon \\
        A &\rightarrow BAB \ |\ 00 \ |\ BB \ |\ AB \ |\ BA \\
        B &\rightarrow 00
    \end{align*}

    \item Removendo a regra unitária $S_0 \rightarrow a$
    \begin{align*}
        S_0 &\rightarrow BAB \ |\ 00 \ |\ BB \ |\ AB \ |\ BA  \ |\ \epsilon \\
        A &\rightarrow BAB \ |\ 00 \ |\ BB \ |\ AB \ |\ BA \\
        B &\rightarrow 00
    \end{align*}

    \item Simplificando, tomando $X \rightarrow AB$ e $Y \rightarrow 0$
    \begin{align*}
        S_0 &\rightarrow BX \ |\ YY \ |\ BB \ |\ AB \ |\ BA  \ |\ \epsilon \\
        A &\rightarrow BX \ |\ YY \ |\ BB \ |\ AB \ |\ BA \\
        B &\rightarrow YY \\
        X &\rightarrow AB \\
        Y &\rightarrow 0
    \end{align*}

\end{enumerate}

\noindent \textbf{L4.5} \\[3pt]
\textbf{Resposta: TODO}

\noindent \textbf{L4.6 Sipser(2.27)}
\begin{enumerate}[label={\textbf{\alph*.}}]
    
\item Mostre que $G$ é uma gramática ambígua.\\[3pt]
\textbf{Resposta: } Podemos mostrar que $G$ é ambígua quando em uma derivação ela produz duas árvores sintáticas distintas para uma mesma cadeia. Seja $w = $ \textbf{if condition then if condition then a := 1 else a := 1}.

\begin{figure}[H]
    \centering

    \begin{tikzpicture}
    \tikzset{frontier/.style={distance from root=200pt,sibling distance=22pt}}
    \Tree [.STMT 
            [.{IF-THEN} \textbf{if} \textbf{condition} \textbf{then}
                [.STMT  [.{IF-THEN-ELSE}
                            \textbf{if} \textbf{condition} \textbf{then}
                            [.STMT [.ASSIGN {\textbf{a := 1}} ] ]
                            \textbf{else}
                            [.STMT [.ASSIGN {\textbf{a := 1}} ] ]
                        ]
                ]
            ]
          ]
    \end{tikzpicture}
\end{figure}

\begin{figure}[H]
    \centering
    
    \begin{tikzpicture}
    \tikzset{frontier/.style={distance from root=200pt,sibling distance=22pt}}
    \Tree [.STMT 
            [.{IF-THEN-ELSE} \textbf{if} \textbf{condition} \textbf{then}
                [.STMT  [.{IF-THEN}
                            \textbf{if} \textbf{condition} \textbf{then}
                            [.STMT [.ASSIGN {\textbf{a := 1}} ] ] 
                        ]
                ]
                \textbf{else}
                [.STMT [.ASSIGN {\textbf{a := 1}} ] ]
            ]
          ]
    \end{tikzpicture}
\end{figure}

\end{enumerate}

\begin{center}
\textbf{\large{Lista 5}}
\end{center}

\noindent \textbf{L5.1} Construa a gramática obtida a partir do autômato a pilha da figura 2.15 usando o lema 2.27, a segunda parte do teorema 2.20. Particione o conjunto de variáveis da sua gramática de forma a exibir quais delas geram a linguagem vazia e quais geram linguagens não vazias.\\[3pt]
\textbf{Resposta: } Referência em \cite{portland}. O autômato dado na figura 2.15 não atende às três condições solicitadas no lema 2.27, a saber:

\begin{enumerate}[label={\textbf{\arabic*.}}]
    \item Deve ter um  único estado de aceitação
    \item A pilha deve estar vazia ao aceitar
    \item Toda transição ou empilha, ou desempilha um símbolo
\end{enumerate}

Claramente o autômato não atende à condição 1. Podemos adicionar uma transição $\epsilon, \epsilon \rightarrow \epsilon$ de $q_1$ a $q_4$ para que $q_1$ deixe de ser um estado de aceitação, porém, o novo autômato não atenderia à condição 3. Logo, vamos adicionar um estado intermediário entre $q_1$ e $q_4$ e transições que empilham e desempilham o símbolo $\texttt{\#}$, respectivamente. Note que essa alteração faz com que o  autômato atenda às 3 condições solicitadas.

\begin{center}
\begin{tikzpicture}[scale=0.2]
\tikzstyle{every node}+=[inner sep=0pt]
\draw [black] (34.3,-14.5) circle (3);
\draw (34.3,-14.5) node {$q_1$};
\draw [black] (18.6,-22.5) circle (3);
\draw (18.6,-22.5) node {$q_5$};
\draw [black] (33.2,-30.4) circle (3);
\draw (33.2,-30.4) node {$q_4$};
\draw [black] (33.2,-30.4) circle (2.4);
\draw [black] (53.3,-14.5) circle (3);
\draw (53.3,-14.5) node {$q_2$};
\draw [black] (53.3,-30.4) circle (3);
\draw (53.3,-30.4) node {$q_3$};
\draw [black] (27.5,-14.5) -- (31.3,-14.5);
\fill [black] (31.3,-14.5) -- (30.5,-14) -- (30.5,-15);
\draw [black] (31.63,-15.86) -- (21.27,-21.14);
\fill [black] (21.27,-21.14) -- (22.21,-21.22) -- (21.76,-20.33);
\draw (19.17,-17.97) node [above] {$\epsilon, \epsilon \rightarrow \texttt{\#}$};
\draw [black] (21.24,-23.93) -- (30.56,-28.97);
\fill [black] (30.56,-28.97) -- (30.1,-28.15) -- (29.62,-29.03);
\draw (18.59,-26.97) node [below] {$\epsilon, \texttt{\#} \rightarrow \epsilon$};
\draw [black] (50.3,-30.4) -- (36.2,-30.4);
\fill [black] (36.2,-30.4) -- (37,-30.9) -- (37,-29.9);
\draw (43.25,-30.9) node [below] {$\epsilon, \$ \rightarrow \epsilon$};
\draw [black] (55.98,-29.077) arc (144:-144:2.25);
\draw (60.55,-30.4) node [right] {$1, 0 \rightarrow \epsilon$};
\fill [black] (55.98,-31.72) -- (56.33,-32.6) -- (56.92,-31.79);
\draw [black] (55.98,-13.177) arc (144:-144:2.25);
\draw (60.55,-14.5) node [right] {$0, \epsilon \rightarrow 0$};
\fill [black] (55.98,-15.82) -- (56.33,-16.7) -- (56.92,-15.89);
\draw [black] (37.3,-14.5) -- (50.3,-14.5);
\fill [black] (50.3,-14.5) -- (49.5,-14) -- (49.5,-15);
\draw (43.8,-14) node [above] {$\epsilon, \epsilon \rightarrow \$$};
\draw [black] (53.3,-17.5) -- (53.3,-27.4);
\fill [black] (53.3,-27.4) -- (53.8,-26.6) -- (52.8,-26.6);
\draw (53.8,-22.45) node [right] {$1, 0 \rightarrow \epsilon$};
\end{tikzpicture}
\end{center}

Seja $Q = \{q_1, q_2, q_3,q_4, q_5\}$ o conjunto de estados do autômato acima modificado. Vamos denominar por $G$ a gramática que vamos construir a partir do autômato de acordo com o procedimento dado no lema 2.27. A regra inicial é:

$S \rightarrow A_{14}$

\begin{enumerate}[label={\textbf{\arabic*.}}]
    \item Para cada $p \in Q$, insira a regra $A_{pp} \rightarrow \epsilon$ em $G$\\[2pt]
    $A_{11} \rightarrow \epsilon$\\
    $A_{22} \rightarrow \epsilon$\\
    $A_{33} \rightarrow \epsilon$\\
    $A_{44} \rightarrow \epsilon$\\
    $A_{55} \rightarrow \epsilon$
    
    \item Para cada $p, q, r \in Q$, insira a regra $A_{pq} \rightarrow A_{pr}A_{rq}$ em $G$\\[2pt]
    $A_{11} \rightarrow A_{11}A_{11}\ |\ A_{12}A_{21}\ |\ A_{13}A_{31}\ |\ A_{14}A_{41}\ |\ A_{15}A_{51}$\\
    $A_{12} \rightarrow A_{11}A_{12}\ |\ A_{12}A_{22}\ |\ A_{13}A_{32}\ |\ A_{14}A_{42}\ |\ A_{15}A_{52}$\\
    $A_{13} \rightarrow A_{11}A_{13}\ |\ A_{12}A_{23}\ |\ A_{13}A_{33}\ |\ A_{14}A_{43}\ |\ A_{15}A_{53}$\\
    $A_{14} \rightarrow A_{11}A_{14}\ |\ A_{12}A_{24}\ |\ A_{13}A_{34}\ |\ A_{14}A_{44}\ |\ A_{15}A_{54}$\\
    $A_{15} \rightarrow A_{11}A_{15}\ |\ A_{12}A_{25}\ |\ A_{13}A_{35}\ |\ A_{14}A_{45}\ |\ A_{15}A_{55}$\\[4pt]
    $A_{21} \rightarrow A_{21}A_{11}\ |\ A_{22}A_{21}\ |\ A_{23}A_{31}\ |\ A_{24}A_{41}\ |\ A_{25}A_{51}$\\
    $A_{22} \rightarrow A_{21}A_{12}\ |\ A_{22}A_{22}\ |\ A_{23}A_{32}\ |\ A_{24}A_{42}\ |\ A_{25}A_{52}$\\
    $A_{23} \rightarrow A_{21}A_{13}\ |\ A_{22}A_{23}\ |\ A_{23}A_{33}\ |\ A_{24}A_{43}\ |\ A_{25}A_{53}$\\
    $A_{24} \rightarrow A_{21}A_{14}\ |\ A_{22}A_{24}\ |\ A_{23}A_{34}\ |\ A_{24}A_{44}\ |\ A_{25}A_{54}$\\
    $A_{25} \rightarrow A_{21}A_{15}\ |\ A_{22}A_{25}\ |\ A_{23}A_{35}\ |\ A_{24}A_{45}\ |\ A_{25}A_{55}$\\[4pt]
    $A_{31} \rightarrow A_{31}A_{11}\ |\ A_{32}A_{21}\ |\ A_{33}A_{31}\ |\ A_{34}A_{41}\ |\ A_{35}A_{51}$\\
    $A_{32} \rightarrow A_{31}A_{12}\ |\ A_{32}A_{22}\ |\ A_{33}A_{32}\ |\ A_{34}A_{42}\ |\ A_{35}A_{52}$\\
    $A_{33} \rightarrow A_{31}A_{13}\ |\ A_{32}A_{23}\ |\ A_{33}A_{33}\ |\ A_{34}A_{43}\ |\ A_{35}A_{53}$\\
    $A_{34} \rightarrow A_{31}A_{14}\ |\ A_{32}A_{24}\ |\ A_{33}A_{34}\ |\ A_{34}A_{44}\ |\ A_{35}A_{54}$\\
    $A_{35} \rightarrow A_{31}A_{15}\ |\ A_{32}A_{25}\ |\ A_{33}A_{35}\ |\ A_{34}A_{45}\ |\ A_{35}A_{55}$\\[4pt]
    $A_{41} \rightarrow A_{41}A_{11}\ |\ A_{42}A_{21}\ |\ A_{43}A_{31}\ |\ A_{44}A_{41}\ |\ A_{45}A_{51}$\\
    $A_{42} \rightarrow A_{41}A_{12}\ |\ A_{42}A_{22}\ |\ A_{43}A_{32}\ |\ A_{44}A_{42}\ |\ A_{45}A_{52}$\\
    $A_{43} \rightarrow A_{41}A_{13}\ |\ A_{42}A_{23}\ |\ A_{43}A_{33}\ |\ A_{44}A_{43}\ |\ A_{45}A_{53}$\\
    $A_{44} \rightarrow A_{41}A_{14}\ |\ A_{42}A_{24}\ |\ A_{43}A_{34}\ |\ A_{44}A_{44}\ |\ A_{45}A_{54}$\\
    $A_{45} \rightarrow A_{41}A_{15}\ |\ A_{42}A_{25}\ |\ A_{43}A_{35}\ |\ A_{44}A_{45}\ |\ A_{45}A_{55}$\\[4pt]
    $A_{51} \rightarrow A_{51}A_{11}\ |\ A_{52}A_{21}\ |\ A_{53}A_{31}\ |\ A_{54}A_{41}\ |\ A_{55}A_{51}$\\
    $A_{52} \rightarrow A_{51}A_{12}\ |\ A_{52}A_{22}\ |\ A_{53}A_{32}\ |\ A_{54}A_{42}\ |\ A_{55}A_{52}$\\
    $A_{53} \rightarrow A_{51}A_{13}\ |\ A_{52}A_{23}\ |\ A_{53}A_{33}\ |\ A_{54}A_{43}\ |\ A_{55}A_{53}$\\
    $A_{54} \rightarrow A_{51}A_{14}\ |\ A_{52}A_{24}\ |\ A_{53}A_{34}\ |\ A_{54}A_{44}\ |\ A_{55}A_{54}$\\
    $A_{55} \rightarrow A_{51}A_{15}\ |\ A_{52}A_{25}\ |\ A_{53}A_{35}\ |\ A_{54}A_{45}\ |\ A_{55}A_{55}$
    
    \item Por fim, para cada $p, q, r, s \in Q$, $t \in \Gamma$ e $a, b \in \Sigma_\epsilon$, insira a regra $A_{pq} \rightarrow aA_{rs}b$ em $G$, se $\delta(p, a, \epsilon)$ contém $(r, t)$ e $\delta(s, b, t)$ contém $(q, \epsilon)$\\[2pt]
    $A_{14} \rightarrow \epsilon A_{23} \epsilon\ |\ \epsilon A_{55} \epsilon$\\
    $A_{23} \rightarrow 0A_{22}1\ |\ 0A_{23}1$
\end{enumerate}

Com isso, concluímos a construção da gramática $G$.\\[6pt]



\noindent \textbf{L5.2 (Sipser 2.20)} Seja $A/B = \{w \ |\ wx \in A$ para algum $x \in B\}$. Mostre que, se $A$ é livre do contexto e $B$ é regular, então $A/B$ é livre do contexto.\\[3pt]
\textbf{Resposta: TODO}


\noindent \textbf{L5.3 (Sipser 2.30)} Use o lema do bombeamento para mostrar que as seguintes linguagens não são livres do contexto.
\begin{enumerate}[label={\textbf{\alph*.}}]
    
    \item $L_a = \{0^n1^n0^n1^n \ |\ n \geq 0\}$\\[3pt]
    \textbf{Resposta:} Vamos usar o lema do bombeamento para mostrar que $L_a$ não é livre do contexto. A prova é por contradição.
    
    Suponha o contrário, ou seja, que $L_a$ é livre do contexto. Seja $p$ o comprimento de bombeamento dado pelo lema do bombeamento. Seja $s$ a cadeia $s = 0^p1^p0^p1^p$. Como $s \in L_a$ e $|s| \geq p$, o lema do bombeamento garante que $s$ pode ser dividida em cinco partes $s = uvxyz$, onde, $\forall i \geq 0$, $uv^ixy^iz \in L_a$. Vamos mostrar que isso é impossível, analisando todas as possibilidades de particionamento de $s$.
    
    Sabemos que $s$ tem a forma:
    \[
    \begin{array}{rrrr|rrrr|rrrr|rrrr}
        \undermat{p}{0&0&0&0} & \undermat{p}{1&1&1&1} & \undermat{p}{0&0&0&0} & \undermat{p}{1&1&1&1} \\
    \end{array}
    \]
    \
    
    Além disso, a condição 3 do lema do bombeamento diz que $|vxy| \leq p$.
    
    \begin{enumerate}[label={\textbf{Caso \arabic*:}}]
        \item $vxy$ não ultrapassa o limite de uma parte de $s$\\[2pt]
        Sem perda de generalidade, vamos considerar que $vxy$ está na primeira parte de $s$. Pelo lema do bombeamento, podemos bombear $v$ e $y$ $i$ vezes $\forall i \geq 0$. Se tomarmos $w = uv^2xy^2z$, claramente teremos mais $0$s na primeira metade e, portanto, $w \notin L_a$, o que é uma contradição.
        
        \item $vxy$ está contido entre a primeira e a segunda parte de $0$s e $1$s em $s$, tal que $u = 0^a$, $v = 0^b$, $x = 0^c1^d$, $y = 1^e$ e $z = 1^f0^p1^p$, onde $a, b, c, d, e$ e $f \geq 0$, $b$ ou $e \neq 0$, $a + b + c = p$ e $d + e + f = p$\\[2pt]
        Se tomarmos $i = 0$, temos que $w = uxz$. Logo, $a + c < p$, assim como $d + f < p$, o que provoca um deslocamento da metade de $w$ à esquerda, desbalanceando a quantidade de $0$s e $1$s da primeira metade em relação à segunda e, portanto, $w \notin L_a$, o que novamente é uma contradição.
        
        Analogamente, este caso cobre a situação em que $vxy$ está contido entre a terceira e a quarta parte de $0$s e $1$s.
        
        \item $vxy$ está contido entre a segunda e a terceira parte de $1$s e $0$s, ou seja, na metade de $s$\\[2pt]
        Como $|vxy| \leq p$, $vxy$ está após a primeira fronteira e antes da terceira fronteira de $s$. Se tomarmos $i = 0$, obtemos como resultado do bombeamento uma cadeia $w = uxz$, onde o tamanho de cada parte de $w$ será $p\ |\ <p\ |\ <p\ |\ p$, respectivamente. Logo, as ocorrências de $0$s e $1$s da primeira metade não correspondem às da segunda e, portanto, $w \notin L_a$, o que também é uma contradição.
    \end{enumerate}
    
    \item $L_b = \{0^n\#0^{2n}\#0^{3n} \ |\ n \geq 0\}$ - Resposta no livro
    
    \item $L_c = \{w\#t \ |\ w$ é uma subcadeia de $t$, onde $w, t \in \{a, b\}^*\}$ - Resposta no livro
    
    \item $L_d = \{t_1\#t_2 \ldots \#t_k \ |\ k \geq 2$, cada $t_i \in \{a, b\}^*$, e $t_i = t_j$, para algum $i\neq j\}$\\[3pt]
    \textbf{Resposta:} Vamos usar o lema do bombeamento para mostrar que $L_d$ não é livre do contexto. A prova é por contradição.
    
    Suponha o contrário, ou seja, que $L_d$ é livre do contexto. Seja $p$ o comprimento de bombeamento dado pelo lema do bombeamento. Seja $s$ a cadeia $s = a^pb^p\#a^pb^p$. Como $s \in L_d$ e $|s| \geq p$, o lema do bombeamento garante que $s$ pode ser dividida em cinco partes $s = uvxyz$, onde, para qualquer $i \geq 0$, $uv^ixy^iz \in L_d$. Vamos mostrar que isso é impossível.
    
    Vamos analisar todas as possibilidades de particionamento de $s$.
    
    \begin{enumerate}[label={\textbf{Caso \arabic*:}}]
        \item $v$ e $y$ contêm apenas $a'$s da primeira parte de $s$, tal que $u = a^l, v = a^m, x = a^n, y = a^q$ e $z = a^rb^p\#a^pb^p$, onde $l, m, n, q, r \geq 0$, $m$ ou $q \neq 0$ e $l + m + n + q + r = p$.
        
        Pelo lema do bombeamento, podemos bombear $v$ e $y$ $i$ vezes, para qualquer $i \geq 0$. Se tomarmos $i = 0$, temos que a cadeia $uxz$ possui menos $a'$s na primeira parte (antes do $\#$) que na segunda parte (após o $\#$), pois como $m$ ou $q \neq 0$, temos que $l + n + r < p$ e, portanto, esta nova cadeia não pertence a $L_d$, o que é uma contradição.
        
        Analogamente, este caso cobre a situação em que $v$ e $y$ contêm apenas $a'$s da segunda parte de $s$.
        
        \item $v$ e $y$ contêm $b'$s da primeira parte de $s$ e não possuem símbolos da segunda parte.
        
        Temos duas possibilidades:
        \begin{itemize}
            \item $u = a^l, v = a^mb^n, x = b^q, y = b^r$ e $z = b^t\#a^pb^p$, onde $l, m, n, q, r, t \geq 0$, $m + n$ ou $r \neq 0$, $l + m = p$, $n + q + r + t = p$ e $|vxy| \leq p$ ou,
            
            \item $u = a^l, v = a^r, x = a^q, y = a^mb^n$ e $z = b^t\#a^pb^p$, onde $l, m, n, q, r, t \geq 0$, $m + n$ ou $r \neq 0$, $l + m + q + r = p$, $n + t = p$ e $|vxy| \leq p$.
        \end{itemize}
        
        Vamos assumir, sem perda de generalidade, que a quantidade de $b'$s de $v$ ou $y$ é maior que zero, caso contrário, voltaríamos ao caso anterior. Dessa forma, para qualquer uma das duas possibilidades, se fizermos um bombeamento de $v$ e $y$ $i$ vezes, para $i = 0$, temos que a cadeia $uxz$ possui menos $b'$s na primeira parte (antes do $\#$) do que na segunda (após o $\#$) e, portanto, esta nova cadeia não pertence a $L_d$, o que é uma contradição.
        
        Analogamente, este caso cobre a situação em que $v$ e $y$ contêm $b'$s da segunda parte de $s$.

        \item $v$ e $y$ contêm $b'$s da primeira parte de $s$ e $a'$s da segunda parte de $s$.
        
        Pela condição 2 do lema do bombeamento, temos que $v$ ou $y$ possuem ao menos um símbolo. Podemos assumir, sem perda de generalidade, que $v$ ou $y$ possui pelo menos um $b$ da primeira parte e pelo menos um $a$ da segunda parte de $s$, caso contrário, cairíamos em um dos casos já abordados previamente. Ao bombearmos $v$ e $y$ $i$ vezes, para $i = 0$, temos que a cadeia $uxz$ possui menos $b'$s na primeira parte do que na segunda e menos $a'$s na segunda parte do que na primeira, o que é uma contradição, já que essa cadeia não pertence a $L_d$.
        
        Vale notar que é impossível que $v$ e $y$ possuam símbolos iguais de partes diferentes da cadeia $s$ pela condição 3 do lema do bombeamento.
    \end{enumerate}

\end{enumerate}

\noindent \textbf{L5.4 (Sipser 2.32)} Seja $\Sigma = \{1, 2, 3, 4\}$ e $C = \{w \in \Sigma^* \ |\ $em $w$, o número de 1s é igual ao número de 2s, e o número de 3s é igual ao número de 4s $\}$. Mostre que $C$ não é livre do contexto.\\[3pt]
\textbf{Resposta: TODO}


\end{comment}

\begin{center}
\textbf{\large{Lista 6}}
\end{center}

\noindent \textbf{L5.1 (Sipser 3.06)} No Teorema 3.21 mostramos que uma linguagem é Turing-reconhecível sse algum
enumerador a enumera. Por que não usamos o seguinte algoritmo mais simples para a direção de ida da prova? Tal qual anteriormente, $s_1, s_2, \ldots $ é uma lista de todas as cadeias em $\Sigma^*$.

$E$ = "Ignore a entrada.
\begin{enumerate}[label={\textbf{\arabic*.}}, leftmargin=1.05in]
\item Repita o que se segue para $i = 1, 2, 3, \ldots$
\item Rode $M$ sobre $s_i$.
\item Se ela aceita, imprima $s_i$."
\end{enumerate}

\textbf{Resposta: } O algoritmo indica que devemos rodar $M$ sobre todas as cadeias possíveis de $\Sigma^*$. A mudança sugerida no enunciado faz com que $M$ execute em uma cadeia $s_i$ por vez e, caso $M$ aceite $s_i$, o enumerador $E$ imprime-na. Ocorre que $M$ pode entrar em \textit{loop} para uma cadeia $s_k$ qualquer e, por conseguinte, nenhuma outra cadeia subsequente a $s_k$ será impressa por $E$ e, portanto, a linguagem de $M$ será diferente do conjunto de cadeias listadas por $E$.\\[6pt]

\noindent \textbf{L6.2 (Sipser 3.08)} Dê descrições a nível de implementação de máquinas de Turing que decidem as linguagens abaixo sobre o alfabeto $\{0, 1\}$.
\begin{enumerate}[label={\textbf{\alph*.}}]
\item \{$w\ |\ w$ contém o mesmo número de 0s e 1s\}\\[3pt]
\textbf{Resposta: } Vamos chamar de $M_a$ a MT que decide a linguagem em \textbf{a.} Note que o primeiro passo deve aceitar a cadeia vazia pois, neste caso, o número de 0s e 1s é igual a zero.\\[3pt]
$M_a = $ "Sobre a cadeia de entrada $w$:
\begin{enumerate}[label={\textbf{\arabic*.}}, leftmargin=1in]
\item Se $w$ é a cadeia vazia, \textit{aceite}, caso contrário, vá para o passo 2.
\item Faça uma varredura na fita e marque o primeiro 0 que ainda não esteja marcado. Se nenhum 0 desmarcado foi encontrado, \textit{rejeite}. Retorne a cabeça para a extremidade esquerda da fita.
\item Faça uma varredura na fita e marque o primeiro 1 que ainda não esteja marcado. Se a varredura não encontrou nenhum 1 desmarcado, \textit{rejeite}. Retorne a cabeça para a extremidade esquerda da fita.
\item Faça uma nova varredura na fita. Se um 0 ou um 1 desmarcado for encontrado, mova a cabeça para a extremidade esquerda da fita e retorne ao passo 2, caso contrário, \textit{aceite}.
\end{enumerate}

\item \{$w\ |\ w$ contém duas vezes mais 0s que 1s\}\\[3pt]
\textbf{Resposta: } Vamos chamar de $M_b$ a MT que decide a linguagem em \textbf{b.} A estratégia utilizada aqui é similar à linguagem do item \textbf{a.}\\[3pt]
$M_b = $ "Sobre a cadeia de entrada $w$:
\begin{enumerate}[label={\textbf{\arabic*.}}, leftmargin=1in]
\item Repita por duas vezes o próximo estágio:
\item \begin{addmargin}[1em]{0em}
Faça uma varredura na fita e marque o primeiro 0 que ainda não esteja marcado. Se nenhum 0 desmarcado foi encontrado, \textit{rejeite}. Retorne a cabeça para a extremidade esquerda da fita.
\end{addmargin}
\item Faça uma varredura na fita e marque o primeiro 1 que ainda não esteja marcado. Se a varredura não encontrou nenhum 1 desmarcado, \textit{rejeite}. Retorne a cabeça para a extremidade esquerda da fita.
\item Faça uma nova varredura na fita. Se um 0 ou um 1 desmarcado for encontrado, mova a cabeça para a extremidade esquerda da fita e retorne ao passo 1, caso contrário, \textit{aceite}.

\begin{comment}
\item Faça uma varredura na fita da esquerda para a direita e marque o primeiro 0 que ainda não esteja marcado. Mova a cabeça para a direita até encontrar o segundo 0 que não esteja marcado e marque-o. Se nenhum ou apenas um 0 desmarcado foi encontrado, \textit{rejeite}. Retorne a cabeça para a extremidade esquerda da fita.
\item Faça uma varredura na fita da esquerda para a direita e marque o primeiro 1 que ainda não esteja marcado. Se a varredura não encontrou nenhum 1 desmarcado, \textit{rejeite}. Retorne a cabeça para a extremidade esquerda da fita.
\item Faça uma nova varredura na fita da esquerda para a direita. Se pelo menos dois 0s desmarcados foram encontrados, retorne ao passo 1, caso contrário, passe ao passo 4. Antes de mover-se para o passo decidido, mova a cabeça para a extremidade esquerda da fita.
\item Novamente, faça uma varredura na fita da esquerda para a direita. Se houver um 1 desmarcado, \textit{rejeite}, senão \textit{aceite}.
\end{comment}

\end{enumerate}

\item \{$w\ |\ w$ \textbf{não} contém duas vezes mais 0s que 1s\}\\[3pt]
\textbf{Resposta: } Vamos chamar de $M_c$ a MT que decide a linguagem em \textbf{c.} A estratégia utilizada aqui é aplicar a MT obtida em \textbf{b.} como uma subrotina.\\[3pt]
$M_c = $ "Sobre a cadeia de entrada $w$:
\begin{enumerate}[label={\textbf{\arabic*.}}, leftmargin=1in]
\item Rode $w$ na máquina $M_b$.
\item Se $M_b$ aceita $w$, \textit{rejeite}, senão, \textit{aceite}.
\end{enumerate}

\end{enumerate}

\noindent \textbf{L6.3 (Sipser 3.14)} Um \textbf{\textit{autômato com fila}} é como um autômato com pilha, exceto que a pilha é substituída por uma fila. Uma \textbf{\textit{fila}} é uma fita que permite que símbolos sejam escritos somente na extremidade esquerda e lidos somente da extremidade direita. Cada operação de escrita (denominá-la-emos  \textit{empurrar}) adiciona um símbolo na extremidade esquerda da fila e cada operação de leitura (denominá-la-emos  \textit{puxar}) lê e remove um símbolo na extremidade direita. Como com um \texttt{AP}, a entrada é colocada numa fita de entrada de somente-leitura separada, e a cabeça sobre a fita de entrada pode mover somente da esquerda para a direita. A fita de entrada contem uma célula com um símbolo em branco após a entrada, de modo que essa extremidade da entrada possa ser detectada. Um autômato com fila aceita sua entrada entrando num estado especial de aceitação em qualquer momento. Mostre que uma linguagem pode ser reconhecida por um autômato com fila determinístico \textbf{sse} a linguagem é Turing-reconhecível.

\textbf{Resposta: } Vamos denominar de \textit{cauda} a extremidade esquerda da fila onde a operação \textit{empurrar} escreve símbolos. Analogamente, vamos denominar de \textit{cabeça} a extremidade direita da fila onde a operação \textit{puxar} lê e remove símbolos.
\vskip 0.2in
\textsc{Afirmação:} Mostre que uma linguagem pode ser reconhecida por um autômato com fila determinístico \textbf{sse} a linguagem é Turing-reconhecível.
\vskip 0.2in
Antes de iniciar a demonstração, vale lembrar que, pela definição 3.5 dada em \cite{sipser:2006}, uma linguagem é \textit{\textbf{Turing-reconhecível}}, se alguma máquina de Turing (MT) a reconhece.

Além disso, vamos ocultar a definição formal de um \textit{\textbf{autômato com fila}}, dado que o enunciado já diz que a única diferença do autômato com pilha é, obviamente, a substituição da pilha pela fila, bem como as operações de leitura/escrita.

\begin{proof}
Para que possamos provar essa afirmação, nós devemos incluir o autômato com fila como uma variante do modelo de MT tal como, MT multifita, MT não determinística, enumeradores que vimos em sala no capítulo 3.2 do livro \cite{sipser:2006}. Logo, devemos mostrar que uma MT e um autômato com fila são equivalentes e, para tanto, temos de demonstrar o seguinte:
\vskip 0.1in
\textbf{(a)} Dada uma máquina de Turing $M$, podemos gerar um autômato com fila determinístico $A$ que reconhece a mesma linguagem de $M$.\\[3pt]
Para provar que podemos gerar $A$ a partir de $M$, temos que mostrar que é possível simular todas as operações de $M$ com o autômato $A$. Seja $a, b \in \Gamma$ de $M$. Logo, temos que simular as transições de $M$ para a direita (D) e esquerda (E) com as operações possíveis em $A$.

Primeiro, vamos escrever um símbolo, digamos \$, para indicar onde está a cabeça da fita de $M$, de forma tal que em $P$ temos $aw_1 \ldots  w_i\$w_{i+1} \ldots w_n$, onde $a$ é o símbolo sob a cabeça da fita de $M$, $w_1 \ldots w_i$ é o conteúdo da fita à direita de $a$ e $w_{i+1} \ldots w_n$ o conteúdo à esquerda de $a$.

$a \rightarrow b, D$

Nesse caso, basta \textit{puxar} $a$ da fila e \textit{empurrar} o símbolo $b$ na \textit{cauda}. Essa operação simula a transição à direita de $M$, uma vez que a cabeça da fita se move para o próximo caractere. Note que, ao encontrar o símbolo $\$$, nós devemos desfazer esse movimento.

$a \rightarrow b, E$

Esse movimento implica que devemos mover o caractere na \textit{cauda} da fila para a \textit{cabeça}.
Seja o conteúdo da fila neste instante como $w_1 \ldots  w_n\$a$ e $a$ o símbolo que vamos movimentar. Primeiro, vamos \textit{empurrar} um símbolo, digamos \texttt{\#} para indicar a posição atual da fita, sendo assim, temos $w_1 \ldots  w_n\$a\texttt{\#}$. Depois, vamos \textit{puxar} e \textit{empurrar} todos os símbolos da fila de $P$ até atingir \texttt{\#} novamente e, então, nós descartamos este caractere pois todo o conteúdo da fila foi devidamente deslocado para a direita, o que resulta em $aw_1 \ldots  w_n\$$. Sendo assim, temos a simulação da transição à esquerda em $M$.
\vskip 0.1in
\textbf{(b)} Dado um autômato com fila determinístico $A$, podemos gerar uma máquina de Turing $M$ que reconhece a mesma linguagem de $A$.\\[3pt]
Nessa direção da prova nós devemos mostrar que $M$ pode simular $A$ ao produzir o mesmo efeito das operações de \textit{empurrar} e \textit{puxar}. Seja $a$ o símbolo que vamos manipular.
\begin{enumerate}
\item No caso de \textit{empurrar}, basta que a cabeça da fita de $M$ percorra a entrada e escreva $a$ ao encontrar o primeiro espaço em branco, representado pelo símbolo \textvisiblespace.
\item Para simular a operação de \textit{puxar}, a cabeça da fita é movida até a extremidade esquerda, onde está o símbolo $\$$, e marca o símbolo $a$ após $\$$ que ainda não foi marcado com uma marca especial, tal como $\ddt{a}$, indicando que ele foi removido.
\end{enumerate}

Dessa forma nós podemos concluir que uma linguagem pode ser reconhecida por um autômato com fila determinístico \textbf{sse} a linguagem é Turing-reconhecível.

\end{proof}

\noindent \textbf{L6.4 (Sipser 3.16)} Mostre que a coleção de linguagens Turing-reconhecíveis é fechada sob a operação de união, concatenação, estrela e intersecção.\\[6pt]
\textbf{Resposta: } Sejam $L_1$ e $L_2$ linguagens Turing-reconhecíveis e $M_1$ e $M_2$ máquinas de Turing que reconhecem $L_1$ e $L_2$, respectivamente. Vamos mostrar que a classe de linguagens Turing-reconhecíveis é fechada sob as seguintes operações:
\begin{enumerate}[label={\textbf{\alph*.}}]
    \item união\\[3pt]
    %%%% New answer
    Vamos construir uma máquina de Turing $M$ capaz de reconhecer a união $L_1 \cup L_2$ da seguinte forma:\\[3pt]
    $M =$ “Sobre a cadeia de entrada $w$:
    \begin{enumerate}[label={\textbf{\arabic*.}}, leftmargin=1in]
        \item Repita o seguinte para $i = 1, 2, 3, \ldots$
    
        \item Rode as máquinas $M_1$ e $M_2$ sobre $w$ por $i$ passos. Se $M_1$ ou $M_2$ aceita $w$, \textit{aceite}. Se $M_1$ e $M_2$ param e rejeitam $w$, então \textit{rejeite}.”

    \end{enumerate}
    
    A máquina construída reconhece cadeias de $L_1 \cup L_2$, pois se $M_1$ ou $M_2$ aceitam $w$, em algum momento a máquina $M$ aceitará $w$, já que a máquina vai, passo a passo, tentando reconhecer $w$ em $M_1$ e $M_2$ simultaneamente. $M$ poderá entrar em \textit{loop} se $M_1$ ou $M_2$ entrar em \textit{loop} e ambas rejeitarem $w$.
    
    \item concatenação\\[3pt]
    %%%% New answer
    Vamos construir uma máquina de Turing $M$ capaz de reconhecer a concatenação $L_1L_2$.
    Seja $w$ uma cadeia de comprimento $n$, tal que pode ser particionada em duas partes. Seja $p_i = x_iy_{n-1}$ cada uma dessas possíveis partições, onde $i = 0, 1, \ldots, n$ representa a quantidade de caracteres de $w$ (a partir do início) na primeira parte da partição e $n – i$ representa a quantidade de caracteres de $w$ (a partir de $i + 1$) na segunda parte da partição. Vamos inserir na fita da máquina $M$ cada uma das $p_i$ partições separadas por um símbolo, digamos $\texttt{\#}$, da forma $\texttt{\#}p_0\texttt{\#}p_1\texttt{\#}\ldots p_n\texttt{\#}$, e cada parte da partição $p_i$ separada por um outro caractere, digamos $\beta$, da forma $x_i\beta y_{n-i}$, ou seja, a fita ficaria da forma $\texttt{\#}x_0\beta y_n\texttt{\#}x_1\beta y_{n-1}\texttt{\#}\ldots \texttt{\#}x_n\beta y_0$. O que faremos é rodar a máquina $M_1$, simultaneamente, em todas as primeiras partes $x_i$ de cada partição $p_i$ e a máquina $M_2$ em todas as segundas partes $y_{n-i}$ de cada partição $p_i$. Se $M_1$ e $M_2$ aceitar alguma partição $p_i$ ($M_1$ aceitando a primeira parte e $M_2$ aceitando a segunda), então $M$ \textit{aceita} $w$. Se para todo $i$, $M_1$ e $M_2$ rejeitar $p_i$ ($M_1$ rejeitando a primeira parte ou $M_2$ rejeitando a segunda), então $M$ \textit{rejeita} $w$. A máquina $M$ pode entrar em \textit{loop} se $w \notin L_1L_2$ e $M_1$ ou $M_2$ entrar em \textit{loop} na tentativa de reconhecimento das partições.
    
    \item estrela\\[3pt]
    %%%% New answer
    Vamos construir uma máquina de Turing $M$ capaz de reconhecer a operação estrela $L_1^*$.
    Seja $w$ uma cadeia de comprimento $n$, tal que pode ser particionada em $m$ partes. Seja $p_i = w'_{i1}w'_{i2} \ldots w'_{im}$ , para $i = 0, 1, \ldots$ cada uma dessas possíveis partições. Vamos inserir na fita da máquina $M$ cada uma das $p_i$ partições separadas por um símbolo, digamos $\texttt{\#}$, da forma $\texttt{\#}p_0\texttt{\#}p_1\texttt{\#} \ldots p_n\texttt{\#}$, e cada parte da partição $p_i$ separada por um outro caractere, digamos $\beta$, da forma $w'_{i1}\beta w'_{i2} \ldots$, ou seja, a fita ficaria da forma $\texttt{\#}w'_{01}\beta w'_{02}\beta \ldots w'_{0m}\texttt{\#}w'_{11}\beta w'_{12}\beta \ldots \texttt{\#}$. O que faremos é rodar a máquina $M_1$ simultaneamente em todas as partes $w'_{i1}w'_{i2}\ldots w'_{im}$ de cada partição $p_i$. Se $M_1$ aceitar alguma partição $p_i$ ($M_1$ aceitando cada uma das partes de $p_i$), então $M$ \textit{aceita} $w$. Se para todo $i$, $M_1$ rejeitar $p_i$, então $M$ \textit{rejeita} $w$. A máquina $M$ pode entrar em \textit{loop} se $w \notin L_1^*$ e $M_1$ entrar em \textit{loop} na tentativa de reconhecimento das partições.
    
    \item intersecção\\[3pt]
    %%%% New answer
    Vamos construir uma máquina de Turing $M$ capaz de reconhecer a intersecção $L_1 \cap L_2$ da seguinte forma:\\[3pt]
    $M =$ “Sobre a cadeia de entrada $w$:
    \begin{enumerate}[label={\textbf{\arabic*.}}, leftmargin=1in]
        \item Repita o seguinte para $i = 1, 2, 3, \ldots$
    
        \item Rode as máquinas $M_1$ e $M_2$ sobre $w$ por $i$ passos. Se $M_1$ e $M_2$ aceitam $w$, \textit{aceite}. Se $M_1$ ou $M_2$ para e rejeita $w$, então \textit{rejeite}.”
    \end{enumerate}
    
    A máquina construída aceita cadeias de $L_1 \cap L_2$, pois se $w \in L_1 \cap L_2$, significa que $\exists w \ |\ w \in L_1$ e $w \in L_2$. Como a máquina testa $w$ em $M_1$ e $M_2$, e só aceita caso $w$ seja aceita por ambas, temos que a máquina $M$ reconhece as cadeias de $L_1 \cap L_2$. $M$ poderá entrar em \textit{loop} se $M_1$ ou $M_2$ entrar em \textit{loop} e rejeitar $w$.
\end{enumerate}

\printbibliography
\end{document}
