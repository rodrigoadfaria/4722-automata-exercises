% --------------------------------------------------------------
% This is all preamble stuff that you don't have to worry about.
% Head down to where it says "Start here"
% --------------------------------------------------------------

\documentclass[12pt]{article}

\usepackage[latin1,utf8]{inputenc}
\usepackage[brazil]{babel}

\usepackage[margin=1in]{geometry}
\usepackage{amsmath,amsthm,amssymb}
\usepackage{clrscode3e} % for the pseudocode
\usepackage{graphicx}
\usepackage{float}
\usepackage{framed}
\usepackage{caption}
% for tables
\usepackage{booktabs}
\usepackage[table,xcdraw]{xcolor}
% for the graphs %
\usepackage{tikz}
\usetikzlibrary{trees}
\usetikzlibrary{shapes,snakes}
\usepackage{verbatim}

\usepackage{tabto}
\usepackage{enumitem}

\graphicspath{ {imgs/} }

\newcommand{\N}{\mathbb{N}}
\newcommand{\Z}{\mathbb{Z}}

% for the theorems, proofs, definitions
%\theoremstyle{plain}
%\newtheorem*{prop}{Afirmação}

\newcommand{\indbase}{\textbf{\textit{\\Base: }}}
\newcommand{\indhypo}{\textbf{\textit{\\Hipótese de Indução: }}}
\newcommand{\indstep}{\textbf{\textit{\\Passo: }}}

\renewcommand\qedsymbol{$\blacksquare$}
\newcommand*\rot{\rotatebox{90}}

\newcommand\floor[1]{\big\lfloor#1\big\rfloor}
\newcommand\ceil[1]{\big\lceil#1\big\rceil}
\newcommand\bgfrac[2]{\bigg(\dfrac{#1}{#2}\bigg)}


\begin{document}

\title{MAC 4722 - Linguagens, Autômatos e Computabilidade}
\author{Rodrigo Augusto Dias Faria - NUSP 9374992\\
Departamento de Ciência da Computação - IME/USP}

\maketitle

%\begin{center}
%\textbf{\large{Lista 1}}
%\end{center}
%
\noindent \textbf{Sipser 0.11} Encontre o erro na seguinte prova de todos os cavalos são da mesma cor.


\textsc{Afirmação:} Em qualquer conjunto de $h$ cavalos, todos os cavalos são da mesma cor.

\begin{proof} Por indução sobre $h$.

\indbase Para $h = 1$. Em qualquer conjunto contendo somente um cavalo, todos os cavalos claramente têm a mesma cor.

\indstep Para $k \geq 1$ assuma que a afirmação seja verdadeira para $h = k$ e prove que ela é verdadeira para $h = k + 1$. Tome qualquer conjunto $H$ de $k + 1$ cavalos. Mostramos que todos os cavalos nesse conjunto são da mesma cor. Remova um cavalo desse conjunto para obter o conjunto $H_1$ com apenas $k$ cavalos. Pela hipótese da indução, todos os cavalos em $H_1$ são da mesma cor.

Agora recoloque o cavalo removido e remova um diferente para obter o conjunto $H_2$. Pelo mesmo argumento, todos os cavalos em $H_2$ são da mesma cor. Por conseguinte, todos os cavalos em $H$ têm que ser da mesma cor, e a prova está
completa.
\end{proof}

\textbf{Resposta:} De fato, a prova por indução está devidamente elaborada, ou seja, apresenta a base, hipótese e passo da indução e, no desenrolar da demonstração, a hipótese é aplicada para provar a afirmação dada.

Ocorre que o argumento é válido para quase todo $h$, porém falha quando $h = 2$, ou seja, quando $k = 1$, vejamos.

Seja $H = \{h_1, h_2\}$. Se removermos $h_2$, teremos um novo conjunto com um único cavalo $H_1 = \{h_1\}$ e, obviamente, todos os cavalos em $H_1$ são da mesma cor. O mesmo vale se retirarmos $h_1$ de $H$, o que nos dá outro conjunto $H_2 = \{h_2\}$. Mas isso é insuficiente para concluir que todos os cavalos em $H$ têm a mesma cor, uma vez que os cavalos em $H_1$ e $H_2$ podem ter cores diferentes.\\[6pt]
%\noindent L1.1 Dado um grafo $G$ sem laços nem arestas múltiplas, dizemos que $G$ é uma árvore se qualquer par de vértices distintos é interligado por um único caminho que não repete vértices. Demonstre que toda Árvore $G$ com pelo menos $n \geq 1$ vértices possui exatamente $n - 1$ arestas. Sugestão: aplique uma indução em $n$; ou suponha por absurdo a existência de um contraexemplo com quantidade mínima de arestas.

\begin{proof}
Prova por indução em $n$.\\

Sejam $V$ e $E$ o conjunto de vértices e arestas do grafo $G$, respectivamente.\\

\textbf{Base:} para $n = 1$, temos:
\begin{align*}
|E[G]| = 0 &= n - 1 \\
      &= 1 - 1 \\
      &= 0
\end{align*}

\textbf{Hipótese de Indução:} assuma que a afirmação é verdadeira para qualquer árvore com um número de vértices menor do que $n$.\\

\textbf{Passo:} vamos considerar uma árvore $T$ com $n$ vértices. Seja $e$ uma aresta que conecta dois vértices $u$ e $v$ em $T$. Por definição de árvore, o único caminho entre $u$ e $v$ é a aresta $e$. Vamos remover $e$. Logo, teremos dois novos componentes (conexos e acíclicos), que também são árvores, $T'$ e $T''$, onde:
\begin{align*}
|V[T']|  &= n'  \text{,}\\
|V[T'']| &= n'' \quad \text{e} \\
n' + n'' &= n
\end{align*}

Ambas $T'$ e $T''$ têm menos vértices que $T$, logo:
\begin{align*}
|E[T']|  &= n' - 1   \quad (\text{por indução}) \\
|E[T'']| &= n'' - 1 \quad (\text{por indução})
\end{align*}

Não é difícil perceber que $T' \cup T''$ possui $(n' - 1) + (n'' - 1) = n -2$ arestas e, portanto, se adicionarmos $e$ de volta, teremos $n - 1$ arestas.

Como queríamos demonstrar!

\end{proof}
%
\noindent \textbf{L1.02} Uma palavra (cadeia) $x$ é dita uma potência de uma palavra $z$ se existir um inteiro $n \geq 0$ tal que $z^n = x$. Duas palavras $x$ e $y$ comutam entre si se $xy = yx$. Prove que duas palavras dadas $x$ e $y$ comutam se e somente se existir uma palavra $z$ da qual $x$ e $y$ são potências. Sugestão: no caso mais difícil, aplique uma indução na soma dos comprimentos de $x$ e $y$.

\textbf{Resposta:} $\Leftarrow$ \textsc{Afirmação:} Se $x = z^i$, $y = z^j$, então $xy = yx \quad \forall i, j \in \N$

Podemos provar essa afirmação da seguinte forma:
\begin{align*}
    xy = z^iz^j = z^{i+j} \\
    yx = z^jz^i = z^{j+i} \\
    \therefore xy = yx
\end{align*}

$\Rightarrow$ \textsc{Afirmação:} Se $xy = yx$, então $\exists z | z^i = x, z^j = y \quad \forall i, j \in \N$

\begin{proof} Prova por indução no tamanho da palavra.

\indbase Para $|x| = 0$ ou $|y| = 0$, temos:

Por definição de $\epsilon$ sabemos que $|\epsilon| = 0$.

No caso em que $|x| = 0$, temos que $x = \epsilon = z^0$. O mesmo vale no caso em que $|y| = 0$.

Se $|x| = |y|$, como sabemos que $xy = yx$, então podemos concluir que $x = y$, pois cada carácter da concatenação do lado esquerdo é exatamente igual ao que está na mesma posição do lado direito.

\indhypo Vamos assumir que a afirmação vale para $|x| < |y|$.

\indstep Neste caso, como sabemos que as cadeias comutam, podemos escrever $y = xv$ onde $v \in z^n \quad \forall n \in \N$. Em outras palavras, $x$ é um prefixo de $y$.

Logo:
\begin{align*}
    xy   &= yx  \\
    xxv  &= xvx \\
    xv   &= vx  \\
    |xv| &< |xy| \quad (\text{por indução})
\end{align*}
\end{proof}

\begin{center}
\textbf{\large{Lista 2}}
\end{center}

\noindent \textbf{L2.1 (Sipser 1.16)} Resolva o exercício 1.16.\\[3pt]
\noindent\textbf{a) Resposta:} Vamos chamar de $M$ o AFN dado na questão. Seja $M_d = \{Q', \Sigma, \delta', q_{0'}, F'\}$ o AFD equivalente à $M$.

\noindent\textit{Estados de $M_d$:} $Q' = \{\{\}, \{1\}, \{2\}, \{1,2\} \}$.

\noindent\textit{Estado inicial:} $q_{0'} = E(\{1\}) = \{1\}$. É o conjunto de estados que são atingíveis a partir de $\{1\}$ viajando por setas $\epsilon$, mais o próprio $\{1\}$.

\noindent\textit{Estados de aceitação:} $F' = \{\{1\}, \{1,2\}\}$. Aqueles que contêm um estado de aceitação de $M$.

\noindent\textit{Função de transição:} $\delta' = $
\begin{table}[!h]
\centering
\rot{\hspace{5 mm}\llap{Estados}}
\begin{tabular}{l|l|l}
         & a         & b        \\ \hline
\{\}     & \{\}      & \{\}     \\
\{1\}    & \{1,2\}   & \{2\}    \\
\{2\}    & \{\}      & \{1\}    \\
\{1,2\}  & \{1,2\}   & \{1,2\}
\end{tabular}
\caption{Função de transição de $M_d$.}\vspace*{0.2cm}
\label{tbl:sip1.16a}
\end{table}

\begin{figure}[!h]
\centering
\begin{tikzpicture}[scale=0.2]
\tikzstyle{every node}+=[inner sep=0pt]
\draw [black] (17.6,-19.6) circle (3);
\draw (17.6,-19.6) node {$\{1\}$};
\draw [black] (17.6,-19.6) circle (2.4);
\draw [black] (47.5,-36) circle (3);
\draw (47.5,-36) node {$\{\}$};
\draw [black] (17.6,-36) circle (3);
\draw (17.6,-36) node {$\{2\}$};
\draw [black] (47.5,-19.6) circle (3);
\draw (47.5,-19.6) node {$\{1,2\}$};
\draw [black] (47.5,-19.6) circle (2.4);
\draw [black] (19.342,-22.033) arc (28.49627:-28.49627:12.088);
\fill [black] (19.34,-33.57) -- (20.16,-33.1) -- (19.28,-32.63);
\draw (21.31,-27.8) node [right] {$b$};
\draw [black] (50.18,-34.677) arc (144:-144:2.25);
\draw (54.75,-36) node [right] {$a,\mbox{ }b$};
\fill [black] (50.18,-37.32) -- (50.53,-38.2) -- (51.12,-37.39);
\draw [black] (20.6,-36) -- (44.5,-36);
\fill [black] (44.5,-36) -- (43.7,-35.5) -- (43.7,-36.5);
\draw (32.55,-35.5) node [above] {$a$};
\draw [black] (15.808,-33.604) arc (-150.49236:-209.50764:11.784);
\fill [black] (15.81,-22) -- (14.98,-22.45) -- (15.85,-22.94);
\draw (13.78,-27.8) node [left] {$b$};
\draw [black] (50.18,-18.277) arc (144:-144:2.25);
\draw (54.75,-19.6) node [right] {$a,\mbox{ }b$};
\fill [black] (50.18,-20.92) -- (50.53,-21.8) -- (51.12,-20.99);
\draw [black] (20.6,-19.6) -- (44.5,-19.6);
\fill [black] (44.5,-19.6) -- (43.7,-19.1) -- (43.7,-20.1);
\draw (32.55,-20.1) node [below] {$a$};
\draw [black] (10.5,-19.6) -- (14.6,-19.6);
\fill [black] (14.6,-19.6) -- (13.8,-19.1) -- (13.8,-20.1);
\end{tikzpicture}
\caption{Diagrama de estados para o AFD $M_d$.}
\label{fig:sip1.16a}
\end{figure}

% item b) %
\noindent\textbf{b) Resposta:} Vamos chamar de $N$ o AFN dado na questão. Seja $N_d = \{Q', \Sigma, \delta', q_{0'}, F'\}$ o AFD equivalente à $N$.

\noindent\textit{Estados de $N_d$:} $Q' = \{\{\}, \{1\}, \{2\}, \{3\}, \{1,2\}, \{1,3\}, \{2,3\}, \{1,2,3\} \}$.

\noindent\textit{Estado inicial:} $q_{0'} = E(\{1\}) = \{1,2\}$.

\noindent\textit{Estados de aceitação:} $F' = \{\{2\}, \{1,2\}, \{2,3\}, \{1,2,3\} \}$. Aqueles que contêm um estado de aceitação de $N$.

\noindent\textit{Função de transição:} $\delta' = $
\begin{table}[!h]
\centering
\rot{\hspace{5 mm}\llap{Estados}}
\begin{tabular}{l|l|l}
            & a             & b         \\ \hline
\{\}        & \{\}          & \{\}      \\
\{1\}       & \{3\}         & \{\}      \\
\{2\}       & \{1,2\}       & \{\}      \\
\{3\}       & \{2\}         & \{2,3\}   \\
\{1,2\}     & \{1,2,3\}     & \{\}      \\
\{1,3\}     & \{2,3\}       & \{2,3\}   \\
\{2,3\}     & \{1,2\}       & \{2,3\}   \\
\{1,2,3\}   & \{1,2,3\}     & \{2,3\}
\end{tabular}
\caption{Função de transição de $N_d$.}\vspace*{0.2cm}
\label{tbl:sip1.16b}
\end{table}

\begin{figure}[!h]
\centering
\begin{tikzpicture}[scale=0.2]
\tikzstyle{every node}+=[inner sep=0pt]
\draw [black] (17.6,-19.6) circle (3);
\draw (17.6,-19.6) node {$\{1,2\}$};
\draw [black] (17.6,-19.6) circle (2.4);
\draw [black] (47.5,-36) circle (3);
\draw (47.5,-36) node {${2,3}$};
\draw [black] (47.5,-36) circle (2.4);
\draw [black] (17.6,-36) circle (3);
\draw (17.6,-36) node {$\{\}$};
\draw [black] (47.5,-19.6) circle (3);
\draw (47.5,-19.6) node {$\{1,2,3\}$};
\draw [black] (47.5,-19.6) circle (2.4);
\draw [black] (50.18,-34.677) arc (144:-144:2.25);
\draw (54.75,-36) node [right] {$b$};
\fill [black] (50.18,-37.32) -- (50.53,-38.2) -- (51.12,-37.39);
\draw [black] (50.18,-18.277) arc (144:-144:2.25);
\draw (54.75,-19.6) node [right] {$a$};
\fill [black] (50.18,-20.92) -- (50.53,-21.8) -- (51.12,-20.99);
\draw [black] (20.6,-19.6) -- (44.5,-19.6);
\fill [black] (44.5,-19.6) -- (43.7,-19.1) -- (43.7,-20.1);
\draw (32.55,-20.1) node [below] {$a$};
\draw [black] (10.5,-19.6) -- (14.6,-19.6);
\fill [black] (14.6,-19.6) -- (13.8,-19.1) -- (13.8,-20.1);
\draw [black] (17.6,-22.6) -- (17.6,-33);
\fill [black] (17.6,-33) -- (18.1,-32.2) -- (17.1,-32.2);
\draw (17.1,-27.8) node [left] {$b$};
\draw [black] (18.923,-38.68) arc (54:-234:2.25);
\draw (17.6,-43.25) node [below] {$a,\mbox{ }b$};
\fill [black] (16.28,-38.68) -- (15.4,-39.03) -- (16.21,-39.62);
\draw [black] (47.5,-22.6) -- (47.5,-33);
\fill [black] (47.5,-33) -- (48,-32.2) -- (47,-32.2);
\draw (47,-27.8) node [left] {$b$};
\draw [black] (44.87,-34.56) -- (20.23,-21.04);
\fill [black] (20.23,-21.04) -- (20.69,-21.87) -- (21.17,-20.99);
\draw (33.49,-27.3) node [above] {$a$};
\end{tikzpicture}
\caption{Diagrama de estados para o AFD $N_d$.}
\label{fig:sip1.16b}
\end{figure}

A figura \ref{fig:sip1.16b} é o AFD simplificado que mostra apenas os estados que são alcançáveis a partir do estado inicial $\{1,2\}$.\\[6pt]

\noindent \textbf{L2.2 (Sipser 1.6c)} Dê um DFA/AFD para $A = \{w | w$ possui $0101$ por subcadeia\}.

\begin{figure}[!h]
\centering
\begin{tikzpicture}[scale=0.2]
    \tikzstyle{every node}+=[inner sep=0pt]
    \draw [black] (7.9,-19.1) circle (3);
    \draw (7.9,-19.1) node {$q_0$};
    \draw [black] (24.5,-19.1) circle (3);
    \draw (24.5,-19.1) node {$q_1$};
    \draw [black] (39.9,-19.1) circle (3);
    \draw (39.9,-19.1) node {$q_2$};
    \draw [black] (55.5,-19.1) circle (3);
    \draw (55.5,-19.1) node {$q_3$};
    \draw [black] (74.1,-19.1) circle (3);
    \draw (74.1,-19.1) node {$q_4$};
    \draw [black] (74.1,-19.1) circle (2.4);
    \draw [black] (0.8,-19.1) -- (4.9,-19.1);
    \fill [black] (4.9,-19.1) -- (4.1,-18.6) -- (4.1,-19.6);
    \draw [black] (10.9,-19.1) -- (21.5,-19.1);
    \fill [black] (21.5,-19.1) -- (20.7,-18.6) -- (20.7,-19.6);
    \draw (16.2,-19.6) node [below] {$0$};
    \draw [black] (27.5,-19.1) -- (36.9,-19.1);
    \fill [black] (36.9,-19.1) -- (36.1,-18.6) -- (36.1,-19.6);
    \draw (32.2,-18.6) node [above] {$1$};
    \draw [black] (42.9,-19.1) -- (52.5,-19.1);
    \fill [black] (52.5,-19.1) -- (51.7,-18.6) -- (51.7,-19.6);
    \draw (47.7,-19.6) node [below] {$0$};
    \draw [black] (58.5,-19.1) -- (71.1,-19.1);
    \fill [black] (71.1,-19.1) -- (70.3,-18.6) -- (70.3,-19.6);
    \draw (64.8,-19.6) node [below] {$1$};
    \draw [black] (6.577,-16.42) arc (234:-54:2.25);
    \draw (7.9,-11.85) node [above] {$1$};
    \fill [black] (9.22,-16.42) -- (10.1,-16.07) -- (9.29,-15.48);
    \draw [black] (23.177,-16.42) arc (234:-54:2.25);
    \draw (24.5,-11.85) node [above] {$0$};
    \fill [black] (25.82,-16.42) -- (26.7,-16.07) -- (25.89,-15.48);
    \draw [black] (26.483,-16.853) arc (134.14834:45.85166:19.407);
    \fill [black] (26.48,-16.85) -- (27.41,-16.65) -- (26.71,-15.94);
    \draw (40,-10.87) node [above] {$0$};
    \draw [black] (37.829,-21.267) arc (-47.84822:-132.15178:20.755);
    \fill [black] (9.97,-21.27) -- (10.23,-22.17) -- (10.9,-21.43);
    \draw (23.9,-27.13) node [below] {$1$};
    \draw [black] (72.777,-16.42) arc (234:-54:2.25);
    \draw (74.1,-11.85) node [above] {$0,\mbox{ }1$};
    \fill [black] (75.42,-16.42) -- (76.3,-16.07) -- (75.49,-15.48);
\end{tikzpicture}
\caption{Diagrama de estados do AFD que reconhece $A$.}
\label{fig:sip1.6c}
\end{figure}


\noindent \textbf{L2.3} Dada uma linguagem $L$, seja $Pref(L) = \{x |$ existe palavra $y$ tal que $xy$ está em $L$\}, $Suf(L) = \{y |$ existe palavra $x$ tal que $xy$ está em $L$\}, $Fat(L) = \{y |$ existem palavras $x$ e $z$ tais que $xyz$ estão em $L$\}.\\
Demonstre que se $L$ é regular, então $Pref(L)$, $Suf(L)$ e $Fat(L)$ também o são. Sugestão: Observe que $Fat(L) = Suf(Pref(L))$.\\[3pt]

\noindent \textbf{L2.4} Complete a demonstração do teorema 1.25.\\[3pt]
\textbf{Resposta:} Vale lembrar, resumidamente, da construção dada na prova do teorema 1.25.

Suponha que $A_1$ e $A_2$ são linguagens reconhecidas por $M_1$ e $M_2$, respectivamente, onde $M_1 = (Q_1, \Sigma, \delta_1, q_1, F_1)$ e $M_2 = (Q_2, \Sigma, \delta_2, q_2, F_2)$.

Construa $M$ para reconhecer $A_1 \cup A_2$, onde $M = (Q, \Sigma, \delta, q_0, F)$.
\begin{enumerate}[label=\textbf{\arabic*}]
    \item $Q = Q_1 \times Q_2$.
    \item $\Sigma$, o alfabeto, é o mesmo em $M_1$ e $M_2$.
    \item $\delta = $ para cada $(r_1, r_2) \in Q$ e cada $a \in \Sigma$, faça $\delta((r_1, r_2), a) = (\delta_1(r_1, a), \delta_2(r_2, a))$.
    \item $q_0 = (q_1, q_2)$.
    \item $F = (F_1 \times Q_2) \cup (Q_1 \times F_2)$.
\end{enumerate}

\begin{proof}
Para demonstrar que $M$ reconhece $A_1 \cup A_2$, devemos dividir a prova em duas partes.

\textsc{Afirmação:} Toda palavra pertencente à linguagem reconhecida por esse autômato está presente em $A_1 \cup A_2$.

Tome uma palavra $w$ qualquer reconhecida pelo autômato $M$. Sabe-se que ao transitarmos através de $\delta$ por $M$, a partir do estado inicial $q_0$, existe um passeio $P$ no autômato $M$ que leva a um estado final. Pela construção de $M$, cada estado nesse passeio é rotulado por um par ordenado $(r_1, r_2)$, onde $r_1 \in M_1$ e $r_2 \in M_2$. Se tomarmos o passeio $P_1$ considerando de $P$ apenas as coordenadas $r_1$ do par ordenado, este é equivalente ao passeio dado pelas transições $\delta_1$ na tentativa de reconhecimento de $w$ em $M_1$. Analogamente, podemos tomar o passeio $P_2$, a partir de $P$, considerando apenas as coordenadas $r_2$, o que equivaleria à tentativa de reconhecimento da palavra $w$ em $M_2$. Pela construção de $M$, temos ainda que o estado final do passeio $P$ é rotulado por um par ordenado $(r_1, r_2)$, onde $r_1 \in F_1$ ou $r_2 \in F_2$. Dessa forma, ou $P_1$ ou $P_2$, ou ambos, terminam com um estado final, logo, $w$ pertence ou a $A_1$, ou a $A_2$, ou a ambas, o que é equivalente a dizer que $w$ pertence à $A_1 \cup A_2$.

\textsc{Afirmação:} Toda palavra pertencente à linguagem $A_1 \cup A_2$ é reconhecida pelo autômato construído.
Tomemos agora uma cadeia $w$ como sendo uma cadeia pertencente a $A_1 \cup A_2$, onde $|w| = m$. Logo, existe um passeio $P_1 = x_0, x_1, \ldots, x_m$ em $M_1$, tal que $x_0 = q_1$ construído a partir de $\delta_1$, ou um passeio $P_2 = z_0, z_1, \ldots, z_m$, construído a partir de $\delta_2$ em $M_2$, tal que $z_0 = q2$, e que $x_m$ ou $z_m$, ou ambos, são estados finais. Como o conjunto de estados $Q$ de $M$ foi construído através do produto cartesiano de $Q_1 \times Q_2$ e a função de transição $\delta((r_1, r_2), a) = (\delta_1(r_1, a), \delta_2(r_2, a))$, para cada par ordenado $(r_1, r_2) \in Q$ e cada $a \in \Sigma$, existe um caminho $P = (x_0, z_0), (x_1, z_1), \ldots, (x_m,z_m)$ em $M$, obtido a partir de $w$, e como $x_m$ ou $z_m$, ou ambos, são estados finais, $(x_m, z_m)$ também é um estado final e, portanto, $M$ reconhece a palavra $w$.
\end{proof}

\end{document}
