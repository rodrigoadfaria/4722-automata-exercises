
\noindent \textbf{L6.2 (Sipser 3.08)} Dê descrições a nível de implementação de máquinas de Turing que decidem as linguagens abaixo sobre o alfabeto $\{0, 1\}$.
\begin{enumerate}[label={\textbf{\alph*.}}]
\item \{$w\ |\ w$ contém o mesmo número de 0s e 1s\}\\[3pt]
\textbf{Resposta: } Vamos chamar de $M_a$ a MT que decide a linguagem em \textbf{a.} Note que o primeiro passo deve aceitar a cadeia vazia pois, neste caso, o número de 0s e 1s é igual a zero.\\[3pt]
$M_a = $ "Sobre a cadeia de entrada $w$:
\begin{enumerate}[label={\textbf{\arabic*.}}, leftmargin=1in]
\item Se $w$ é a cadeia vazia, \textit{aceite}, caso contrário, vá para o passo 2.
\item Faça uma varredura em $w$ da esquerda para a direita e marque o primeiro 0 que ainda não esteja marcado. Se nenhum 0 desmarcado foi encontrado, \textit{rejeite}. Retorne a cabeça para a extremidade esquerda da fita.
\item Faça uma varredura em $w$ da esquerda para a direita e marque o primeiro 1 que ainda não esteja marcado. Se a varredura não encontrou nenhum 1 desmarcado, \textit{rejeite}. Retorne a cabeça para a extremidade esquerda da fita.
\item Faça uma nova varredura em $w$ da esquerda para a direita. Se um 0 desmarcado for encontrado, volte a cabeça uma posição à esquerda e retorne ao passo 2, caso contrário, mova a cabeça para a extremidade esquerda da fita e passe ao passo 5.
\item Novamente, faça uma varredura em $w$ da esquerda para a direita. Se houver um 1 desmarcado, \textit{rejeite}, senão \textit{aceite}.
\end{enumerate}

\item \{$w\ |\ w$ contém duas vezes mais 0s que 1s\}\\[3pt]
\textbf{Resposta: } Vamos chamar de $M_b$ a MT que decide a linguagem em \textbf{b.} A estratégia utilizada aqui é similar à linguagem do item \textbf{a.}\\[3pt]
$M_b = $ "Sobre a cadeia de entrada $w$:
\begin{enumerate}[label={\textbf{\arabic*.}}, leftmargin=1in]
\item Faça uma varredura em $w$ da esquerda para a direita e marque o primeiro 0 que ainda não esteja marcado. Mova a cabeça para a direita até encontrar o segundo 0 que não esteja marcado e marque-o. Se nenhum ou apenas um 0 desmarcado foi encontrado, \textit{rejeite}. Retorne a cabeça para a extremidade esquerda da fita.
\item Faça uma varredura em $w$ da esquerda para a direita e marque o primeiro 1 que ainda não esteja marcado. Se a varredura não encontrou nenhum 1 desmarcado, \textit{rejeite}. Retorne a cabeça para a extremidade esquerda da fita.
\item Faça uma nova varredura em $w$ da esquerda para a direita. Se pelo menos dois 0s desmarcados foram encontrados, retorne ao passo 1, caso contrário, passe ao passo 4. Antes de mover-se para o passo decidido, mova a cabeça para a extremidade esquerda da fita.
\item Novamente, faça uma varredura em $w$ da esquerda para a direita. Se houver um 1 desmarcado, \textit{rejeite}, senão \textit{aceite}.
\end{enumerate}

\item \{$w\ |\ w$ não contém duas vezes mais 0s que 1s\}\\[3pt]
\textbf{Resposta: } Vamos chamar de $M_c$ a MT que decide a linguagem em \textbf{c.} A estratégia utilizada aqui é aplicar a MT obtida em \textbf{b.} como uma subrotina.\\[3pt]
$M_c = $ "Sobre a cadeia de entrada $w$:
\begin{enumerate}[label={\textbf{\arabic*.}}, leftmargin=1in]
\item Rode $w$ na máquina $M_b$.
\item Se $M_b$ aceita $w$, \textit{rejeite}, senão \textit{aceite}.
\end{enumerate}
\end{enumerate}