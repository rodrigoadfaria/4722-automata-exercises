
\noindent \textbf{Sipser 0.11} Encontre o erro na seguinte prova de todos os cavalos são da mesma cor.


\textsc{Afirmação:} Em qualquer conjunto de $h$ cavalos, todos os cavalos são da mesma cor.

\begin{proof} Por indução sobre $h$.

\indbase Para $h = 1$. Em qualquer conjunto contendo somente um cavalo, todos os cavalos claramente têm a mesma cor.

\indstep Para $k \geq 1$ assuma que a afirmação seja verdadeira para $h = k$ e prove que ela é verdadeira para $h = k + 1$. Tome qualquer conjunto $H$ de $k + 1$ cavalos. Mostramos que todos os cavalos nesse conjunto são da mesma cor. Remova um cavalo desse conjunto para obter o conjunto $H_1$ com apenas $k$ cavalos. Pela hipótese da indução, todos os cavalos em $H_1$ são da mesma cor.

Agora recoloque o cavalo removido e remova um diferente para obter o conjunto $H_2$. Pelo mesmo argumento, todos os cavalos em $H_2$ são da mesma cor. Por conseguinte, todos os cavalos em $H$ têm que ser da mesma cor, e a prova está
completa.
\end{proof}

\textbf{Resposta:} De fato, a prova por indução está devidamente elaborada, ou seja, apresenta a base, hipótese e passo da indução e, no desenrolar da demonstração, a hipótese é aplicada para provar a afirmação dada.

Ocorre que o argumento é válido para quase todo $h$, porém falha quando $h = 2$, ou seja, quando $k = 1$, vejamos.

Seja $H = \{h_1, h_2\}$. Se removermos $h_2$, teremos um novo conjunto com um único cavalo $H_1 = \{h_1\}$ e, obviamente, todos os cavalos em $H_1$ são da mesma cor. O mesmo vale se retirarmos $h_1$ de $H$, o que nos dá outro conjunto $H_2 = \{h_2\}$. Mas isso é insuficiente para concluir que todos os cavalos em $H$ têm a mesma cor, uma vez que os cavalos em $H_1$ e $H_2$ podem ter cores diferentes.\\[6pt]